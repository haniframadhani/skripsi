%==================================================================
% Ini adalah bab 1
% Silahkan edit sesuai kebutuhan, baik menambah atau mengurangi \section, \subsection
%==================================================================

\chapter[PENDAHULUAN]{\\ PENDAHULUAN}

\section{Latar Belakang}
Metaheuristik adalah istilah yang digunakan untuk menyebutkan optimasi stokastik. Optimalisasi stokastik adalah kelas umum algoritma dan teknik yang menggunakan tingkat keacakan tertentu untuk menemukan solusi optimal (atau seoptimal mungkin) terhadap masalah yang sulit \citep{Luke2013Metaheuristics}. Algoritma metaheuristik mengalami pertumbuhan yang pesat. Pada tahun 2022 ada lebih dari 500 algoritma metaheuristik \citep{Rajwar_2023}, \citep{Ma_wu_suganthan_song_luo_2023} yang telah dikembangkan. \textit{Particle Swarm Optimization} (PSO) \citep{Kennedy_1995}, \textit{Genetic Algorithms} (GA) \citep{Mitchell_1998}, \textit{Simulated Annealing} (SA) \citep{Kirkpatrick_1983},\textit{Differential Evolution} (DE) \citep{Price2004DifferentialE} adalah contoh algoritma yang banyak disitasi berdasarkan Google scholar \citep{Rajwar_2023}.

Menurut teori \textit{No Free Launch} (NFL), tidak ada satu algoritma pun yang dapat dianggap sebagai yang terbaik untuk setiap permasalahan. Setiap algoritma memiliki kekuatan dan kelemahan dalam menyelesaikan masalah tertentu. Oleh karena itu, penting untuk memilih algoritma yang paling cocok untuk setiap masalah yang dihadapi. Untuk menentukan algoritma yang sesuai untuk suatu tugas tertentu, pengujian dilakukan menggunakan fungsi-fungsi uji yang dirancang khusus dengan karakteristik yang sesuai dengan masalah yang dihadapi. Berbagai fungsi uji terkumpul ke dalam kerangka kerja seperti \textit{Congress of Evolutionary Computation}(CEC) \citep{pro_1999} dan \textit{Comparing Continuous Optimizers} (COCO) \citep{hansen2021coco}. Kedua kerangka kerja tersebut digunakan secara luas untuk menguji kinerja metaheuristik dengan berbagai fungsi yang telah tersedia. Meskipun memiliki tujuan dan fungsi uji yang serupa, CEC dan COCO memiliki perbedaan dalam pendekatan dan ruang pencarian. CEC lebih berfokus pada kompetisi metaheuristik dengan tujuan optimasi tunggal menggunakan parameter nyata, sementara COCO bertujuan untuk memberikan pemahaman tentang kinerja algoritma dan kondisi di mana algoritma dapat bekerja dengan efektif \citep{_kvorc_2020}.

Saat ini, ada beberapa \textit{library} uji fungsi yang tersedia untuk bahasa pemrograman Python seperti OPFUNU \citep{Van_Thieu_2024}, landscape, dan UQTestFuns, yang merupakan kumpulan fungsi uji dari CEC. Sedangkan \textit{library} yang digunakan untuk COCO adalah COCO \citep{pro_1999}. Kedua jenis \textit{library} ini memainkan peran penting dalam pengembangan algoritma metaheuristik. Menggunakan \textit{library} yang berbeda untuk tujuan yang serupa dapat memperlambat pengembangan algoritma metaheuristik karena berbagai \textit{library} tersebut memiliki cara implementasi yang berbeda, yang dapat menyebabkan konflik dalam proses implementasi. Namun, saat ini keterbatasan \textit{library} tunggal yang mencakup semua fungsi uji dari CEC dan COCO.\ Akibatnya, pengembangan algoritma metaheuristik sering kali mengalami hambatan karena pengujian dengan fungsi-fungsi uji dari CEC atau COCO harus dilakukan secara manual atau dengan menggunakan \textit{library} CEC dan COCO yang terpisah. Memasang beberapa \textit{library} secara bersamaan untuk mengatasi keterbatasan satu \textit{library} juga dapat menambah kompleksitas proyek dan berpotensi mengurangi kinerja keseluruhan proyek.

Saat ini, sangat terbatas sekali \textit{library} yang menggabungkan CEC dan COCO menjadi satu kesatuan. Oleh karena itu, penelitian ini bertujuan untuk mengembangkan \textit{library} yang menggabungkan keduanya. Diharapkan dengan adanya \textit{library} ini, para pengembang algoritma metaheuristik dapat lebih mudah dalam mengembangkan algoritma mereka, sehingga proses pengembangannya pun menjadi lebih cepat dan efisien.

\section{Rumusan Masalah Penelitian}
Berdasarkan latar belakang masalah yang telah dipaparkan sebelumnya, masalah yang dapat dirumuskan adalah bagaimana membangun \textit{library} uji fungsi yang mengintegrasikan kerangka kerja uji fungsi CEC dan COCO ke dalam satu \textit{library} tunggal.

% \section{Batasan Masalah}
% Penelitian ini membatasi perhatian pada dua kerangka kerja pengujian metaheuristik yang umum diimplementasikan dalam lingkup penelitian metaheuristik, yaitu \textit{Congress of Evolutionary Computation} (CEC) dan \textit{Comparing Continuous Optimizers} (COCO).

\section{Tujuan Penelitian}
Tujuan yang ingin dicapai melalui penelitian ini adalah:
\begin{packed_enum}
  \item Mengidentifikasi fungsi uji CEC dan COCO
  \item Membangun \textit{library benchmark} yang berasal dari kerangka kerja CEC dan COCO
  \item Menguji dan mengevaluasi \textit{library} yang dihasilkan
\end{packed_enum}

\section{Manfaat Penelitian}
Manfaat dari penelitian ini adalah memudahkan para peneliti metaheuristik dalam melakukan pengujian algoritma metaheuristik dengan kedua kerangka kerja pengujian yang umum digunakan yaitu \textit{Congress of Evolutionary Computation} (CEC) dan \textit{Comparing Continuous Optimizers} (COCO).