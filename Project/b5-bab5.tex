%==================================================================
% Ini adalah bab 5
% Silahkan edit sesuai kebutuhan, baik menambah atau mengurangi \section, \subsection
%==================================================================

\chapter[KESIMPULAN DAN SARAN]{\\ KESIMPULAN DAN SARAN}

\section{Kesimpulan}
Penelitian ini telah berhasil mengembangkan sebuah \textit{library} Python yang mengintegrasikan berbagai fungsi \textit{benchmark} dari dua kerangka kerja utama dalam evaluasi algoritma optimasi, yakni \textit{Congress on Evolutionary Computation} (CEC) dan \textit{COmparing Continuous Optimizers} (COCO), ke dalam satu platform terpadu. Integrasi ini memberikan kemudahan bagi para peneliti dan praktisi dalam melakukan pengujian algoritma metaheuristik secara efisien tanpa perlu berpindah antar \textit{library} yang berbeda. \textit{Library} ini dirancang dengan pendekatan modular dan prinsip \textit{object-oriented programming} (OOP), yang memungkinkan perluasan fitur, pemeliharaan kode, dan pengujian sistem dilakukan dengan lebih terstruktur. Setiap fungsi \textit{benchmark} diimplementasikan dalam kelas tersendiri yang diturunkan dari kelas induk \textit{Benchmark}, sehingga menjamin konsistensi struktur dan kemudahan dalam penggunaannya. Untuk dokumentasi, \textit{library} ini memanfaatkan Sphinx, yang memungkinkan pembuatan dokumentasi teknis secara otomatis berdasarkan docstring. Dokumentasi tersebut mencakup penjelasan rinci mengenai setiap fungsi, panduan instalasi, serta contoh penggunaan, sehingga mendukung keberlanjutan dan kemudahan adopsi \textit{library} oleh komunitas pengguna.

\section{Saran}
Sebagai tindak lanjut dari penelitian ini, disarankan agar pengembangan \textit{library} CECO diperluas ke berbagai bahasa pemrograman lain, seperti MATLAB, C, Java, dan bahasa lainnya. Diversifikasi platform ini bertujuan untuk meningkatkan fleksibilitas dan memperluas jangkauan pengguna, sehingga \textit{library} tidak hanya terbatas pada ekosistem Python yang saat ini menjadi basis utama pengembangannya. Dengan demikian, CECO dapat diakses dan dimanfaatkan oleh komunitas ilmiah yang lebih luas dan beragam, termasuk peneliti dan praktisi yang bekerja dalam lingkungan non-Python.

Selain perluasan platform, perlu juga dilakukan pengayaan terhadap koleksi fungsi \textit{benchmark} yang tersedia dalam \textit{library} . Penambahan variasi fungsi uji ini akan memungkinkan evaluasi algoritma optimasi dilakukan secara lebih komprehensif, mencakup berbagai jenis karakteristik permasalahan optimasi, baik dari segi unimodalitas, multimodalitas, hingga kompleksitas dimensi. Dengan cakupan \textit{benchmark} yang lebih luas, hasil evaluasi algoritma akan menjadi lebih representatif dan relevan.

Hasil pengujian sebelumnya menunjukkan adanya nilai invalid yang muncul akibat \textit{overflow} pada tipe data float64 maupun longdouble di NumPy, terutama ketika fungsi \textit{benchmark} seperti Shubert diuji dalam dimensi yang sangat tinggi. Fenomena ini mengindikasikan perlunya peningkatan stabilitas numerik dalam implementasi \textit{library}. Oleh karena itu, penelitian selanjutnya dapat difokuskan pada pengembangan strategi mitigasi terhadap masalah numerik, seperti \textit{overflow}, \textit{underflow}, atau ketidakakuratan akibat keterbatasan presisi data tipe \textit{floating point}. Pendekatan yang dapat diterapkan antara lain mencakup penggunaan tipe data dengan presisi lebih tinggi, penerapan teknik normalisasi input, serta penyusunan ulang formulasi matematis untuk menjaga kestabilan komputasi. Upaya-upaya ini diharapkan mampu meningkatkan keandalan dan akurasi perhitungan, serta memperluas penerapan CECO pada kasus-kasus ekstrem atau kompleks secara numerik.
