%==================================================================
% Ini adalah abstrak dalam bahasa indonesia 
%==================================================================

%% DILARANG EDIT BAGIAN INI
\clearpage
\phantomsection
\addcontentsline{toc}{chapter}{ABSTRAK}
\begin{center}
    \large{\textbf{ABSTRAK}}\\[1cm]
\end{center}
%% DILARANG EDIT BAGIAN INI

%% edit bagian ini
Algoritma metaheuristik merupakan teknik optimasi stokastik yang memanfaatkan keacakan untuk menemukan solusi optimal pada masalah kompleks, dengan lebih dari 500 algoritma telah dikembangkan hingga 2022. Setiap algoritma baru memerlukan tahap evaluasi melalui fungsi uji standar sebelum digunakan secara luas, namun proses ini seringkali memakan waktu karena peneliti harus menuliskan fungsi uji secara manual ke dalam kode sumber. Dua kerangka kerja yang paling umum digunakan untuk pengujian adalah \textit{Congress on Evolutionary Computation} (CEC) dan \textit{Comparing Continuous Optimizers} (COCO), namun keduanya tersedia secara terpisah, sehingga menyulitkan proses \textit{benchmarking} algoritma secara efisien dan terstandardisasi. Untuk mengatasi hal ini, penelitian ini bertujuan mengembangkan sebuah \textit{library} Python bernama CECO, yang mengintegrasikan kedua kerangka kerja tersebut ke dalam satu platform terpadu, sehingga mempercepat dan mempermudah pengujian algoritma metaheuristik.

Pengembangan CECO dilakukan menggunakan bahasa Python dengan pendekatan modular berbasis Pemrograman Berorientasi Objek (OOP). CECO menyediakan berbagai fungsi \textit{benchmark} dari kerangka CEC dan COCO, serta menangani berbagai permasalahan nyata dengan fleksibilitas tinggi sehingga pengguna dapat melakukan kustomisasi sesuai kebutuhan spesifik penelitian. Proses pengembangan meliputi perencanaan desain, implementasi kode dengan dokumentasi menggunakan Sphinx, serta pengujian menggunakan \textit{unit testing} untuk memastikan keandalan fungsi. Struktur folder terorganisasi memudahkan pengguna menjalankan eksperimen secara fleksibel, serta memungkinkan integrasi dengan \textit{pipeline} eksperimen dan pengembangan algoritma optimasi yang ada.

Hasil implementasi menunjukkan CECO mampu menjalankan lebih dari 30 fungsi \textit{benchmark} dari CEC dan COCO secara terpadu, mempermudah akses dan penggunaan fungsi uji secara komprehensif dan terstandarisasi. CECO mempercepat proses pengembangan algoritma metaheuristik, meningkatkan kualitas evaluasi, dan mengurangi kompleksitas konfigurasi saat menggunakan \textit{library benchmarking} secara terpisah. Dengan demikian, CECO menjadi alat bantu yang efektif, praktis, dan efisien bagi peneliti dan praktisi untuk menguji dan mengembangkan inovasi algoritma optimasi secara menyeluruh. \textit{Library} CECO tersedia secara gratis di GitHub, memungkinkan siapa pun untuk memanfaatkan dan berkontribusi pada pengembangannya demi kemajuan riset optimasi global.\\[0.6cm]
%% edit sampai sini

%% DILARANG EDIT BAGIAN INI
\noindent Kata kunci: \textit{Benchmark}; CEC; COCO; \textit{Metaheuristic}; \textit{Optimization}; Python
%% DILARANG EDIT BAGIAN INI