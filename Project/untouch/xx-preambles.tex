%==================================================================
% Konfigurasi utama untuk dokumen LaTeX
%==================================================================

%% DILARANG EDIT BAGIAN INI

% Mengatur bahasa dokumen ke Bahasa Indonesia dan encoding karakter
\usepackage[english,indonesian]{babel}
% \usepackage[utf8]{inputenc}
\usepackage[autostyle]{csquotes}

% Pengaturan jarak antar baris dan penyesuaian kotak teks agar rata kiri
\usepackage{setspace}
\usepackage[raggedrightboxes]{ragged2e}

% Paket untuk memasukkan gambar dalam dokumen
\usepackage{graphicx}
\graphicspath{{gambar/}} % Menentukan folder default untuk gambar
\usepackage{float}       % Mengatur posisi gambar dalam teks

% Mengatur indentasi paragraf
\usepackage{indentfirst}  % Memastikan paragraf pertama di setiap section memiliki indentasi
\setlength\parindent{1cm} % Mengatur jarak indentasi paragraf menjadi 1 cm

% Mengatur font utama dokumen menjadi Calibri
\usepackage[no-math]{fontspec}
\setmainfont{Calibri}
\setmonofont{Courier New}
% \setmathfont{Latin Modern Math}

% Mengatur format penomoran section dan subsection
\renewcommand{\thesection}{\arabic{chapter}.\arabic{section}}
\renewcommand{\thesubsection}{\thesection.\arabic{subsection}}
\renewcommand{\thesubsubsection}{\thesubsection.\arabic{subsubsection}}

% Pengaturan daftar isi, daftar gambar, dan daftar tabel
\usepackage{tocloft}
\cftsetpnumwidth{1.5em}
\cftsetrmarg{2.5em}
\setlength{\cftsecnumwidth}{3.5em}
\setlength{\cftsubsecnumwidth}{3.5em}
\setlength{\cftsubsubsecnumwidth}{3.5em}
\renewcommand{\cftchapdotsep}{\cftdotsep}
\setlength{\cftbeforechapskip}{3pt}
\renewcommand{\cftchapfont}{\normalfont}
\renewcommand{\cftchapleader}{\normalfont\cftdotfill{\cftdotsep}}
\renewcommand{\cftchappagefont}{\normalfont}

% Penyesuaian judul bab dalam daftar isi agar ditampilkan sebagai "BAB"
\renewcommand\cftchappresnum{BAB }
\renewcommand\cftchapaftersnum{}
\newlength\mylen
\settowidth\mylen{\bfseries BAB 1 :\ } % Menyesuaikan lebar penomoran bab
\cftsetindents{chap}{0pt}{\mylen}

% Pengaturan format dan penomoran judul bab
\usepackage{titlesec}
\titleformat{\chapter}{\doublespacing\fontsize{14pt}{16pt}\bfseries}{\MakeUppercase{\chaptertitlename\ \Roman{chapter}}\filcenter}{0.15cm}{\centering\uppercase}
\titleformat{\section}{\fontsize{12}{14}\bfseries}{\thesection}{0.4cm}{}
\titleformat{\subsection}{\fontsize{12}{14}\bfseries}{\thesubsection}{0.675cm}{}
\titleformat{\subsubsection}{\fontsize{12}{14}\bfseries}{\thesubsubsection}{0.675cm}{}
\setcounter{tocdepth}{3}
\setcounter{secnumdepth}{3}

% Mengatur jarak dan format spacing untuk chapter, section dan subsection
\titlespacing*{\chapter}{0pt}{-1cm}{20pt}
\titlespacing*{\section}{0pt}{10pt}{0cm}
\titlespacing*{\subsection}{0pt}{10pt}{0cm}
\titlespacing*{\subsubsection}{0pt}{10pt}{0cm}

% Pengaturan untuk caption gambar dan tabel
\usepackage[font=small, format=plain, labelfont=normalfont, up, textfont=up]{caption}
\usepackage{subcaption}

% Menghapus tanda titik dua pada caption
\captionsetup[figure]{labelsep=space}
\captionsetup[table]{labelsep=space}

% Mengatur nomor caption gambar dan tabel sesuai bab
\renewcommand{\thefigure}{\arabic{chapter}.\arabic{figure}}
\renewcommand{\thetable}{\arabic{chapter}.\arabic{table}}

% Mengatur hyphenation (pemisahan kata) agar lebih rapi
\tolerance=1
\emergencystretch=\maxdimen
\hyphenpenalty=10000
\hbadness=10000

% Pengaturan tabel, multirow, dan ukuran kolom otomatis
\usepackage{tabularx}
\usepackage{multirow}

% Pengaturan margin halaman
\usepackage{geometry}
\geometry{
    left=3cm,          % Margin kiri
    top=3cm,           % Margin atas
    right=3cm,         % Margin kanan
    bottom=3cm,        % Margin bawah
}

% Paket untuk notasi matematika
\usepackage{amsmath}
\usepackage{amssymb}

% Pengaturan nomor pada persamaan matematika sesuai bab
\renewcommand{\theequation}{\arabic{chapter}.\arabic{equation}}
\makeatletter
\renewcommand{\theequation}{\arabic{chapter}.\arabic{equation}}
\renewcommand{\@eqnnum}{\theequation}
% \def\tagform@#1{\maketag@@@{#1}} % This line removes the parentheses
\makeatother

% Untuk halaman berorientasi landscape
\usepackage{rotating}

% Pengaturan untuk list (daftar item dan angka)
\usepackage{enumitem}
\setlist{nosep} % Menghilangkan jarak antar item dalam list
\newenvironment{packed_enum}{ % Membuat lingkungan untuk daftar bernomor
    \begin{enumerate}[leftmargin=1.5\parindent]
        \setlength{\itemsep}{0pt}
        \setlength{\parskip}{0pt}
        \setlength{\parsep}{0pt}
        }{\end{enumerate}}

\newenvironment{packed_item}{ % Membuat lingkungan untuk daftar berpoin
    \begin{itemize}[leftmargin=1.375\parindent]
        \setlength{\itemsep}{0pt}
        \setlength{\parskip}{0pt}
        \setlength{\parsep}{0pt}
        }{\end{itemize}}

% Paket untuk bibliografi menggunakan BibTeX
\usepackage[numbers]{natbib}
\bibliographystyle{ieeetr}

% Paket untuk tabel yang panjang dan melampaui satu halaman
\usepackage{longtable}

% Paket untuk memasukkan hyperlink dalam dokumen
\usepackage{hyperref}

% Pengaturan label otomatis untuk berbagai elemen (gambar, tabel, persamaan)
\usepackage{cleveref}
\crefname{figure}{gambar}{gambar}
\Crefname{figure}{Gambar}{Gambar}
\crefname{table}{tabel}{tabel}
\Crefname{table}{Tabel}{Tabel}
\crefformat{equation}{persamaan~#2#1#3}
\crefname{equation}{persamaan}{persamaan}
\Crefname{equation}{Persamaan}{Persamaan}
\crefname{lstlisting}{kode}{kode}
\Crefname{lstlisting}{Kode}{Kode}

%\AtBeginDocument{\renewcommand{\lstlistingname}{Kode}} 
%\AtBeginDocument{\renewcommand{\thelstlisting}{\thechapter.\arabic{lstlisting}}}
%\AtBeginDocument{\renewcommand{\thelstlisting}{\arabic{chapter}.\arabic{lstlisting}.}}
\AtBeginDocument{\renewcommand{\lstlistingname}{Kode}}
\AtBeginDocument{\renewcommand{\thelstlisting}{\arabic{chapter}.\arabic{lstlisting}}}

\captionsetup[lstlisting]{
  format=plain,
  labelfont=normalfont,
  justification=centering,
  singlelinecheck=false,
  labelsep=space
}

% Paket untuk menampilkan kode program
\usepackage{listings}
\lstdefinestyle{newstyle}{
    backgroundcolor=,                      % Tidak ada warna latar
    commentstyle=,                         % Tidak ada pewarnaan komentar
    keywordstyle=,                         % Tidak ada pewarnaan keyword
    stringstyle=,                          % Tidak ada pewarnaan string
    basicstyle=\ttfamily\fontsize{10}{12}\selectfont,  % Courier New ukuran 10pt
    columns=fullflexible,
    breakatwhitespace=false,
    breaklines=true,
    captionpos=b,
    keepspaces=true,
    numbers=left,                          % Nomor baris di sebelah kiri
    numbersep=5pt,                         % Border di semua sisi
    showspaces=false,
    showstringspaces=false,
    showtabs=false,
    tabsize=2,
    lineskip=-1pt,
    xleftmargin=2em,
    frame=single,
    framexleftmargin=2em
}
\lstset{style=newstyle}

%paket untuk gambar dengan tikz
\usepackage{tikz}
\usepackage{pgfplots}
\usepackage{pgf-pie}
\pgfplotsset{compat=1.18}
\usetikzlibrary{shapes.geometric, arrows}
\tikzstyle{startstop} = [rectangle, rounded corners, minimum width=3cm, minimum height=1cm,text centered, draw=black, fill=red!30]
\tikzstyle{process} = [rectangle, minimum width=3cm, minimum height=1cm, text centered, draw=black, fill=orange!30]
\tikzstyle{decision} = [diamond, minimum width=3cm, minimum height=1cm, text centered, draw=black, fill=green!30]
\tikzstyle{arrow} = [thick,->,>=stealth]

%untuk if-then
\usepackage{ifthen}

\usepackage{eso-pic}

\newcommand\BackgroundPic{
    \put(0,0){
        \parbox[b][\paperheight]{\paperwidth}{
            \vfill
            \centering
            \includegraphics[width=4in,keepaspectratio]{gambar/uad-kuning.png}
            \vfill
        }
    }
}

\usepackage{lscape}
\usepackage{adjustbox}

\usepackage{bm}

\DeclareMathOperator{\sign}{sign}

%% DILARANG EDIT BAGIAN INI
