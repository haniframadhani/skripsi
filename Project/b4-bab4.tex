%==================================================================
% Ini adalah bab 4
% Silahkan edit sesuai kebutuhan, baik menambah atau mengurangi \section, \subsection
%==================================================================

\chapter[HASIL DAN PEMBAHASAN]{\\ HASIL DAN PEMBAHASAN}

\section{Implementasi fungsi \textit{benchmark}}
Fungsi-fungsi \textit{benchmark} yang telah dikumpulkan dan dijelaskan pada bagian sebelumnya kemudian diimplementasikan ke dalam kode program menggunakan bahasa pemrograman Python. Proses implementasi dilakukan dengan memperhatikan struktur modular agar memudahkan pemeliharaan, pengujian, dan pengembangan lebih lanjut. Dalam pengembangannya, beberapa dependensi utama digunakan untuk mendukung fungsionalitas dan keandalan \text{library}.\ \text{Library} NumPy digunakan sebagai basis dalam pengolahan data numerik, mengingat kemampuannya dalam menangani operasi matematis dan array berdimensi tinggi secara efisien. Untuk keperluan pengujian fungsionalitas dan validasi unit dari setiap fungsi \textit{benchmark}, digunakan unittest, modul bawaan Python yang mendukung penerapan prinsip pengujian berbasis test case secara sistematis. Sementara itu, dokumentasi teknis dari \text{library} disusun menggunakan Sphinx, sebuah generator dokumentasi yang mampu menghasilkan dokumentasi dalam berbagai format seperti HTML dan PDF, sehingga memudahkan pengguna dan pengembang dalam memahami struktur serta cara penggunaan \text{library}. Dengan kombinasi alat bantu ini, proses pengembangan \text{library} CECO menjadi lebih terstruktur, terdokumentasi, dan dapat diandalkan untuk digunakan dalam berbagai eksperimen optimasi.

\subsection{\textit{Super Class}}
Kelas induk (\textit{superclass/parent class}) dalam sistem ini diberi nama \textit{Benchmark}, yang berperan sebagai basis untuk seluruh kelas turunan (\textit{child classes}). Kelas ini dirancang dengan pendekatan modular untuk mengintegrasikan berbagai fungsi dasar yang bersifat generik dan sering digunakan berulang kali oleh kelas-kelas turunannya.\\
\subsubsection{Konstruktor}
\textit{Method} ini merupakan konstruktor utama dalam kelas Benchmark. Fungsi utamanya adalah untuk menerima parameter dimensi yang akan digunakan dalam proses evaluasi. Dimensi tersebut menjadi dasar perhitungan di seluruh kelas turunannya dalam implementasi fungsi CEC dan COCO. Syarat dimensi yang dimasukkan harus berupa bilangan bulat positif (\text{integer}) dengan nilai lebih besar dari 0. Selain itu, nilai dimensi tidak diperbolehkan berupa bilangan desimal (\textit{float/double}) maupun bilangan negatif, karena dapat mengganggu proses komputasi dan menghasilkan output yang tidak valid.
\begin{lstlisting}[language=Python, caption=kontruktor kelas benchmark, label=lst:init_benchmark]
    def __init__(self, dimension: int) -> None:
        """
        Initializes the Benchmark class with a rotation matrix and a shift vector.

        Parameters:
            dimension (int): The number of dimensions for the input space. Must be a positive integer.

        Raises:
            ValueError: If `dimension` is not a positive integer.
        """
        if not isinstance(dimension, int):
            raise ValueError("dimension must be integer")
        if dimension < 0:
            raise ValueError("dimension must be positive integer")
        if dimension < 1:
            raise ValueError("dimension cannot be zero")
        self.dimension = dimension
\end{lstlisting}
\subsubsection{cec\_init}
\textit{Method} ini digunakan untuk menerima tiga parameter utama dalam fungsi CEC, yaitu \textit{shift, rotation, dan bias}. Parameter \textit{shift} berupa \textit{array} satu dimensi dengan jumlah elemen yang sesuai dengan nilai dimensi, berfungsi untuk menggeser posisi optimum. \textit{Rotation} berupa matriks dua dimensi berukuran dimensi$\times$dimensi, digunakan untuk merotasi ruang pencarian. \textit{Bias} dapat berupa bilangan bulat atau desimal yang berfungsi sebagai konstanta penambah pada nilai fungsi. Ukuran shift dan rotation harus sesuai dengan dimensi yang telah ditentukan.
\begin{lstlisting}[language=Python, caption=\textit{method} untuk tambahan konstruktor untuk fungsi CEC, label=lst:init_cec]
    def cec_init(self, rotation: np.ndarray, shift: np.ndarray, f_bias: float = 0) -> None:
        """
        Initializes the cec function class with a rotation matrix and a shift vector.

        Parameters:
            rotation (np.ndarray): The rotation matrix.
            shift (np.ndarray): The shift vector.
            f_bias (float): A bias term added to the benchmark function's output. Defaults to 0.
        """
        if not isinstance(rotation, np.ndarray) or rotation.ndim != 2:
            raise ValueError(
                "Rotation matrix must be a non-empty np.ndarray")
        if rotation.shape[0] != self.dimension and shift.shape[0] != self.dimension:
            raise ValueError(
                "rotation dimension and shift dimension does not match dimension")
        elif rotation.shape[0] != self.dimension:
            raise ValueError(
                "rotation dimension does not match dimension")
        elif shift.shape[0] != self.dimension:
            raise ValueError(
                "shift dimension does not match dimension")
        if rotation.shape[0] != rotation.shape[1]:
            raise ValueError("Rotation matrix must be a square matrix")
        if rotation.shape[0] != shift.shape[0]:
            raise ValueError("rotation and shift has different dimensions")
        self.rotation = rotation
        self.shift = shift
        if not isinstance(f_bias, (float, int)):
            raise ValueError(
                "f_bias must be float or int")
        self.f_bias = float(f_bias)
\end{lstlisting}
\subsubsection{create\_diagonal\_matrix}
\textit{Method} ini berfungsi untuk menghasilkan matriks diagonal khusus dengan nilai diagonal berupa pangkat dari parameter alpha ($\alpha $) untuk notasi $\Lambda^{\alpha}$. Matriks tersebut dipakai untuk melakukan transformasi \textit{conditioning} guna memperbesar pengaruh dimensi tertentu dibandingkan dimensi lainnya.
\begin{lstlisting}[language=Python, caption=\textit{method} untuk menghasilkan matriks diagonal, label=lst:init_cec]
    def create_diagonal_matrix(self, alpha: float) -> np.ndarray:
        """
        Creates a diagonal matrix Λ^α, where the diagonal elements are powers of α.

        The diagonal elements are defined as:
            Λ_ii = alpha^(0.5 * (i - 1) / (D - 1)) for i = 1, 2, ..., D,
            where D is the dimension.

        Parameters:
            alpha (float): The base value for the diagonal elements.

        Returns:
            np.ndarray: A diagonal matrix of shape (D, D), where D is the dimension.

        Notes:
            - If the dimension is 1, the matrix is [[1]].
            - If alpha is 0, the matrix is a zero matrix.
        """
        if self.dimension == 1:
            return np.array([[1.0]])

        if alpha == 0:
            return np.zeros((self.dimension, self.dimension))

        exponents = 0.5 * np.arange(self.dimension) / (self.dimension - 1)
        diagonal = alpha ** exponents
        return np.diag(diagonal)
\end{lstlisting}
\subsubsection{generate\_random\_matrix}
\textit{Method} ini berfungsi menghasilkan matriks dua dimensi berukuran dimensi$\times$dimensi secara acak, yang nantinya akan digunakan sebagai nilai matriks $R$ dan $Q$.
\begin{lstlisting}[language=Python, caption=\textit{method} untuk menghasilkan matriks dengan nilai acak, label=lst:init_cec]
    def generate_random_matrix(self, D: int) -> np.ndarray:
        """
        Generates a random matrix with elements drawn from a standard normal distribution.

        Parameters:
            D (int): The size of the matrix (D x D).

        Returns:
            np.ndarray: A random matrix of shape (D, D).
        """
        return np.random.randn(D, D)
\end{lstlisting}
\subsubsection{T\_asy\_beta}
\textit{Method} ini berfungsi untuk melakukan transformasi asimetris pada setiap elemen input. Ini berarti akan mengubah input dengan cara yang tidak seragam, di mana perubahan yang diterapkan pada satu sisi (atau arah) mungkin berbeda dengan sisi lainnya.
\begin{lstlisting}[language=Python, caption=\textit{method} untuk menerapkan transformasi asimetris pada elemen input, label=lst:init_cec]
    def T_asy_beta(self, beta: float, input_vector: np.ndarray) -> np.ndarray:
        """
        Applies the T_asy^beta transformation to a given input vector.

        The transformation is defined as:
            T_asy^beta(x_i) = x_i^(1 + beta * (i / (D - 1)) * sqrt(x_i)) if x_i > 0,
            T_asy^beta(x_i) = x_i otherwise.

        Parameters:
            beta (float): The beta parameter controlling the transformation.
            input_vector (np.ndarray): A vector of real numbers.

        Returns:
            np.ndarray: The transformed vector.

        Notes:
            - If the dimension is 1, the input vector is returned unchanged.
        """
        if self.dimension == 1:
            return input_vector

        i = np.arange(self.dimension)
        mask = input_vector > 0
        exponent = 1 + beta * (i / (self.dimension - 1)
                               ) * np.sqrt(input_vector)
        result = np.where(mask, input_vector ** exponent, input_vector)
        return result
\end{lstlisting}
\subsubsection{T\_osz}
\textit{Method} ini berfungsi untuk menerapkan transformasi non-linear pada elemen input, yang menghasilkan efek osilasi dan ketidakteraturan pada struktur fungsi. Transformasi ini meningkatkan kompleksitas permukaan fungsi, sehingga membuatnya lebih sulit untuk dioptimalkan.
\begin{lstlisting}[language=Python, caption=\textit{method} untuk menerapkan transformasi non-linear pada elemen input, label=lst:init_cec]
    def T_osz(self, input_vector: np.ndarray) -> np.ndarray:
        """
        Applies the T_osz transformation to a given input vector.

        The transformation is defined as:
            T_osz(x_i) = sign(x_i) * exp(x_hat + 0.049 * (sin(c1 * x_hat) + sin(c2 * x_hat))),
            where x_hat = log(|x_i|) if x_i ≠ 0, and 0 otherwise.

        Parameters:
            input_vector (np.ndarray): A vector of real numbers.

        Returns:
            np.ndarray: The transformed vector.

        Notes:
            - The transformation is applied element-wise.
            - If x_i is 0, the result is 0.
        """
        x_hat = np.where(input_vector == 0, 0, np.log(np.abs(input_vector)))
        sign = np.sign(input_vector)
        c1 = np.where(input_vector > 0, 10, 5.5)
        c2 = np.where(input_vector > 0, 7.9, 3.1)
        transformed = np.exp(
            x_hat + 0.049 * (np.sin(c1 * x_hat) + np.sin(c2 * x_hat)))
        return np.where(input_vector == 0, 0, sign * transformed)
\end{lstlisting}
\subsubsection{f\_pen}
\textit{Method} ini berfungsi untuk menerapkan sistem penalti pada elemen input yang melampaui rentang $[-5,5]$. Penalti ini bertujuan mengurangi kemungkinan solusi berada di luar batas tersebut dengan memberikan nilai penalti yang mempengaruhi kualitas solusi. Mekanisme ini berfungsi sebagai pembatas yang mendorong algoritma optimisasi tetap mengeksplorasi dalam rentang yang valid.
\begin{lstlisting}[language=Python, caption=\textit{method} untuk menerapkan penalti pada elemen input, label=lst:init_cec]
    def f_pen(self, x: np.ndarray) -> float:
        """
        Computes the penalty function for a given input vector.

        The penalty function is defined as:
            f_pen(x) = sum(max(0, |x_i| - 5)^2).

        Parameters:
            x (np.ndarray): A vector of real numbers.

        Returns:
            float: The penalty value.
        """
        return np.sum(np.maximum(0, np.abs(x) - 5) ** 2)
\end{lstlisting}
\subsubsection{gram\_schmidt}
\textit{Method} ini berfungsi melakukan proses ortonormalisasi menggunakan algoritma Gram-Schmidt terhadap matriks rotasi beserta elemen-elemen $R$ dan $Q$. Proses ortonormalisasi ini bertujuan untuk memastikan bahwa matriks yang dihasilkan memiliki sifat ortogonal dan ternormalisasi, dimana setiap vektor kolom atau baris saling tegak lurus dan memiliki panjang satuan. Penerapan algoritma Gram-Schmidt ini penting untuk menjaga stabilitas numerik dan validitas matematis dari transformasi rotasi yang akan digunakan dalam fungsi benchmark optimisasi.
\begin{lstlisting}[language=Python, caption=\textit{method} untuk menerapkan penalti pada elemen input, label=lst:init_cec]
    def gram_schmidt(self, matrix: np.ndarray) -> np.ndarray:
        """
        Applies the Gram-Schmidt process to a set of vectors.

        This method orthogonalizes the input vectors (columns of the matrix) and normalizes them to unit length.

        Parameters:
            matrix (np.ndarray): A 2D array where each column is a vector.

        Returns:
            np.ndarray: A 2D array with orthogonal and normalized vectors as columns.

        Notes:
            - If the input vectors are linearly dependent, the output will contain zero vectors for dependent columns.
            - Uses floating-point arithmetic, so results may have small numerical errors.
        """
        matrix = np.array(matrix, dtype=np.float64)
        Q = np.zeros_like(matrix)

        for j in range(matrix.shape[1]):
            v = matrix[:, j]
            for i in range(j):
                v -= np.dot(Q[:, i], v) * Q[:, i]
            norm = np.linalg.norm(v)
            Q[:, j] = v / norm if norm > 1e-14 else 0

        return Q
\end{lstlisting}
\subsection{fungsi CEC dan COCO}
Fungsi-fungsi matematis CEC dan COCO yang telah dikumpulkan dan diidentifikasi sebelumnya diimplementasikan ke dalam sistem dengan pendekatan \textit{Object-Oriented Programming} (OOP) yang ketat. Setiap fungsi CEC dan COCO dikemas dalam kelas terpisah yang mengikuti prinsip \foreignlanguage{english}{\enquote{\textit{one class per file}}}, suatu \textit{best practice} dalam pengembangan perangkat lunak modern yang memberikan beberapa keuntungan penting:
\begin{packed_enum}
	\item Modularitas Tinggi:
    \begin{packed_enum}
      \item Setiap file hanya berisi satu kelas fungsi CEC tertentu
      \item Memisahkan concern dengan jelas antar berbagai fungsi \textit{benchmark}
      \item Memudahkan navigasi dan pemeliharaan kode
    \end{packed_enum}
	\item Struktur Hierarkis:
    \begin{packed_enum}
      \item Seluruh kelas fungsi CEC merupakan turunan (\textit{inheritance}) dari kelas \textit{Benchmark}
      \item Mewarisi semua fungsi dasar dan kemampuan transformasi dari \textit{parent class} 
      \item Mengimplementasikan \textit{interface} yang konsisten untuk semua fungsi \textit{benchmark}
    \end{packed_enum}
	\item Keuntungan Implementasi:
    \begin{packed_enum}
      \item Isolasi perubahan (modifikasi satu fungsi tidak mempengaruhi fungsi lain)
      \item Kemudahan dalam menambahkan fungsi baru tanpa mengganggu sistem yang ada 
      \item Pengelolaan dependensi yang lebih terstruktur
      \item Pengujian unit yang lebih terfokus dan independen
    \end{packed_enum}
	\item Organisasi Kode
	\item Penyimpanan dalam struktur direktori yang jelas
	\item Dokumentasi terintegrasi dalam setiap \textit{file} kelas
\end{packed_enum}
Pendekatan ini memungkinkan:
\begin{packed_enum}
	\item Skalabilitas sistem yang baik untuk penambahan fungsi di masa depan
	\item Kemudahan kolaborasi antar \textit{developer} 
	\item Kemampuan pelacakan perubahan yang lebih baik melalui \textit{version control}
	\item Integrasi yang lebih hasul dengan sistem testing otomatis
\end{packed_enum}
\subsubsection{Konstruktor}
\textit{Method} ini merupakan konstruktor utama dalam kelas fungsi CEC dan COCO. Fungsi utamanya adalah untuk menerima parameter dimensi yang akan digunakan dalam proses evaluasi. Dimensi tersebut menjadi dasar perhitungan di fungsi tersebut. Syarat dimensi yang dimasukkan harus berupa bilangan bulat positif (\text{integer}) dengan nilai lebih besar dari 0. Selain itu, nilai dimensi tidak diperbolehkan berupa bilangan desimal (\textit{float/double}) maupun bilangan negatif, karena dapat mengganggu proses komputasi dan menghasilkan output yang tidak valid. Untuk fungsi CEC menerima parameter \textit{shift} berupa \textit{array} satu dimensi dengan jumlah elemen yang sesuai dengan nilai dimensi, berfungsi untuk menggeser posisi optimum. \textit{Rotation} berupa matriks dua dimensi berukuran dimensi$\times$dimensi, digunakan untuk merotasi ruang pencarian. \textit{Bias} dapat berupa bilangan bulat atau desimal yang berfungsi sebagai konstanta penambah pada nilai fungsi. Ukuran shift dan rotation harus sesuai dengan dimensi yang telah ditentukan.
\begin{lstlisting}[language=Python, caption=kontruktor kelas fungsi CEC, label=lst:init_cec]
    def __init__(self, dimension: int, rotation: np.ndarray, shift: np.ndarray, f_bias: float = 0) -> None:
        """
        Initializes the Sphere class with a rotation matrix and a shift vector.

        Parameters:
            dimension (int): The number of dimensions for the input space. Must be a positive integer.
            rotation (np.ndarray): The rotation matrix for transforming the input vector.
            shift (np.ndarray): The shift vector for adjusting the input vector.
            f_bias (float): A bias term added to the benchmark function's output. Defaults to 0.
        """
        super().__init__(dimension)
        self.cec_init(rotation, shift, f_bias)
\end{lstlisting}
Sementara itu, pada fungsi COCO, beberapa variasi fungsi juga menerima parameter opsional seperti \textit{high conditioning}, \textit{adequate global structure}, dan \textit{ill conditioned} dalam bentuk \textit{boolean}, karena masing-masing fungsi dapat memiliki versi dengan atau tanpa karakteristik tersebut.
\begin{lstlisting}[language=Python, caption=kontruktor kelas fungsi COCO, label=lst:init_cec]
    def __init__(self, dimension: int) -> None:
        """
        Initializes the Sphere function with a given dimension.

        Parameters:
            dimension (int): The number of dimensions for the input space. Must be a positive integer.

        Raises:
            ValueError: If dimension is not a positive integer.
        """
        if not isinstance(dimension, int) or dimension <= 0:
            raise ValueError("Dimension must be a positive integer")
        super().__init__(dimension)
        # Generate a random optimal solution vector x_opt within the range [-5, 5]
        self.x_opt = np.random.uniform(-5, 5, dimension)

        self.f_opt = self.raw(self.x_opt)
\end{lstlisting}
Konstruktor pada fungsi COCO juga menginisialisasi properti $\mathrm{x}^{\text{opt}}$ dan $f_{\text{opt}}$ sebagai titik dan nilai optimum. Selain itu, pada beberapa fungsi, konstruktor turut menginisialisasi properti seperti $R$, $Q$, $\Lambda^{\alpha}$, dan $s_i$ yang digunakan dalam proses transformasi input. Di samping itu, beberapa di fungsi juga menghitung dan menyimpan hasil perhitungan ke dalam variabel di awal untuk menghemat waktu komputasi selama evaluasi fungsi berlangsung.
\begin{lstlisting}[language=Python, caption=kontruktor kelas fungsi COCO, label=lst:init_coco]
    def __init__(self, dimension: int) -> None:
        """
        Initializes the Katsuura function with a given dimension.

        Parameters:
            dimension (int): The number of dimensions for the input space. Must be a positive integer.
            high_conditioning (bool, optional): If `True`, applies a Gram-Schmidt transformation to increase problem conditioning. Defaults to `False`.

        Raises:
            ValueError: If dimension is not a positive integer.
        """
        if not isinstance(dimension, int) or dimension <= 0:
            raise ValueError("Dimension must be a positive integer")
        super().__init__(dimension)
        self.powers_of_two = 2 ** np.arange(1, 33)
        self.inverse_powers = 1 / self.powers_of_two
        self.exponent = 10 / (self.dimension ** 1.2)
        self.scale_factor = 10 / (self.dimension ** 2)
        # Generate a random optimal solution vector x_opt within the range [-5, 5]
        self.x_opt = np.random.uniform(-5, 5, dimension)

        self.f_opt = self.raw(self.x_opt)

        self.R = self.generate_random_matrix(dimension)
        self.R = self.gram_schmidt(self.R)
        self.Q = self.generate_random_matrix(dimension)
        self.Q = self.gram_schmidt(self.Q)
        self.diagonal_matrix = self.create_diagonal_matrix(100)
\end{lstlisting}
\subsubsection{raw}
\textit{Method} ini adalah fungsi dasar yang beroperasi dalam bentuk paling sederhana tanpa melibatkan transformasi tambahan seperti transformasi input, $f_{\text{opt}}$, atau $f_{\text{pen}}$. Fungsi dasar ini berfungsi sebagai komponen inti yang akan dimanfaatkan untuk dua tujuan utama: menghitung nilai $f_{\text{opt}}$ dan melakukan evaluasi pada fungsi-fungsi yang memerlukan transformasi input. Dengan demikian, metode ini berperan sebagai fondasi perhitungan yang kemudian dapat dikembangkan dengan berbagai lapisan transformasi sesuai kebutuhan spesifik fungsi \textit{benchmark}.
\begin{lstlisting}[language=Python, caption=\textit{method raw} fungsi Attractive sector, label=lst:raw_coco]
    def raw(self, x: np.ndarray) -> float:
        """
        Evaluates the Attractive sector function at a given input vector without any shift.

        The function is calculated as:
            f(x) = T_{text{osz}}( sum_{i=1}^{D}( s_iz_i )^2 )^{0.9}

        Parameters:
            x (np.ndarray): A vector of real numbers representing a candidate solution.

        Returns:
            float: The function value at the given input vector.

        Raises:
            ValueError: If the input vector does not match the expected dimension.
        """
        if x.shape[0] != self.dimension:
            raise ValueError(
                f"Input vector must have {self.dimension} elements")

        s = np.where((x * self.x_opt), 10 ** 2, 1)

        result = np.sum((s * x) ** 2)
        return result ** 0.9
\end{lstlisting}
\subsubsection{evaluate}
\textit{Method} ini berfungsi mengevaluasi fungsi pada CEC dan COCO dengan menerapkan berbagai transformasi yang telah didefinisikan. Parameter input yang diterima berupa struktur array yang memiliki jumlah elemen sama dengan nilai dimensi yang telah ditetapkan saat konstruktor kelas diinisialisasi. Konsistensi antara ukuran input dan dimensi yang telah ditentukan merupakan persyaratan wajib untuk memastikan proses evaluasi berjalan dengan benar dan menghasilkan output yang valid.
\begin{lstlisting}[language=Python, caption=\textit{method evaluate} fungsi sphere, label=lst:evaluate_cec]
    def evaluate(self, input_vector: np.ndarray) -> float:
        """
        Evaluates the rotated and shifted sphere function at a given input vector.

        Parameters:
            input_vector (np.ndarray): The input vector [x1, x2, ..., xD].

        Returns:
            float: The result of the sphere function after applying rotation
            and shift.
        """
        # Apply shift and rotation
        shifted_rotated_vector = np.matmul(
            self.rotation, input_vector - self.shift)

        # Calculate the sum of squares
        result = np.sum(shifted_rotated_vector ** 2)

        return result + self.f_bias
\end{lstlisting}
\begin{lstlisting}[language=Python, caption=\textit{method evaluate} fungsi Attractive sector, label=lst:evaluate_coco]
    def evaluate(self, input_vector: np.ndarray) -> float:
        """
        Evaluates the Attractive sector function at a given input vector.

        The function is calculated as:
            f(x) = T_{text{osz}}( sum_{i=1}^{D}( s_iz_i )^2 )^{0.9}+ f_opt

        Parameters:
            input_vector (np.ndarray): A vector of real numbers representing a candidate solution. Must have the same length as the dimension of the Attractive sector function.

        Returns:
            float: The function value at the given input vector.

        Raises:
            ValueError: If the input vector does not match the expected dimension.
        """
        if input_vector.shape[0] != self.dimension:
            raise ValueError(
                f"Input vector must have {self.dimension} elements")

        z = self.Q @ self.diag_matrix @ self.R @ (input_vector - self.x_opt)
        result = self.T_osz(self.raw(z)) + self.f_opt

        return result
\end{lstlisting}
\subsection{Dokumentasi}
Sphinx digunakan sebagai alat dokumentasi standar dalam pengembangan \textit{library} Python CECO. Sistem dokumentasi ini menerapkan pendekatan terintegrasi dimana docstring ditulis langsung dalam \textit{source code}, khususnya pada setiap kelas dan fungsi yang merepresentasikan fungsi \textit{benchmark}. Setiap fungsi \textit{benchmark} memiliki penjelasan komprehensif yang mencakup deskripsi fungsionalitas, parameter yang digunakan, serta detail implementasi yang ditulis dalam kelas yang bersangkutan.

Contoh implementasi docstring dapat dilihat pada kelas Ackley yang ditunjukkan dalam \cref{lst:docstring}. Docstring pada kelas ini memberikan deskripsi lengkap tentang fungsi Ackley sebagai benchmark optimisasi, termasuk formula matematisnya yang kompleks dan multimodal. Dokumentasi juga menjelaskan bahwa kelas ini mewarisi dari kelas Benchmark dan menerapkan transformasi rotasi dan shift pada vektor input sebelum evaluasi fungsi. Setiap atribut seperti dimensi, \textit{rotation}, \textit{shift}, dan $f_\text{bias}$ dijelaskan dengan detail tipe data dan fungsinya.

Metode konstruktor juga didokumentasikan dengan jelas, menjelaskan parameter yang diterima dan validasi yang dilakukan untuk memastikan konsistensi antara matriks rotasi dan vektor \textit{shift}. Sphinx kemudian menggunakan docstring ini untuk secara otomatis menghasilkan dokumentasi dalam berbagai format seperti HTML atau PDF, memastikan dokumentasi selalu terkini dan selaras dengan implementasi kode.
\begin{lstlisting}[language=Python, caption=Dokumentasi menggunakan docstring pada fungsi Ackley, label=lst:docstring]
class Ackley(Benchmark):
    """
    A class representing the Ackley function, which is a benchmark function for optimization.

    The Ackley function is commonly used to test optimization algorithms due to its complex and multi-modal nature. The function is defined as:

    F(X) = -20 * exp(-0.2 * sqrt((1/D) * sum_{i=1}^D x_i^2)) - exp((1/D) * sum_{i=1}^D cos(2 * pi * x_i)) + 20 + e,

    where D is the dimensionality of the problem, and e is Euler's number (~2.71828).

    The class inherits from the `Benchmark` class and applies rotation and shift transformations to the input vector before evaluating the function.

    Attributes:
        dimension (int): The number of dimensions for the input space. Must be a positive integer.
        rotation (np.ndarray): A rotation matrix for transforming the input vector.
        shift (np.ndarray): A shift vector for adjusting the input vector.
        f_bias (float): A bias term added to the benchmark function's output. Defaults to 0.
    """

    def __init__(self, dimension: int, rotation: np.ndarray, shift: np.ndarray, f_bias: float = 0) -> None:
        """
        Initializes the Ackley class with a rotation matrix and a shift vector.

        Ensures that the rotation matrix and shift vector have the same length. If their lengths do not match, a ValueError is raised.

        Parameters:
            dimension (int): The number of dimensions for the input space. Must be a positive integer.
            rotation (np.ndarray): The rotation matrix for transforming the input vector.
            shift (np.ndarray): The shift vector for adjusting the input vector.
            f_bias (float): A bias term added to the benchmark function's output. Defaults to 0.
        """
        super().__init__(dimension)
        self.cec_init(rotation, shift, f_bias)
\end{lstlisting}

Untuk dokumentasi yang tidak terkait langsung dengan fungsi benchmark seperti panduan instalasi, struktur direktori, dan contoh penggunaan \textit{library} CECO menggunakan format reStructuredText (.rst). Contoh \cref{lst:rst} menunjukkan bagaimana modul-modul dalam \textit{package} ceco.cec didokumentasikan menggunakan direktif \textit{automodule} yang secara otomatis mengekstrak \textit{members}, \textit{members} yang tidak memiliki dokumentasi docstring atau doc-comment, dan hierarki pewarisan dari setiap modul.
\begin{lstlisting}[caption=Dokumentasi dengan \textit{reStructuredText}, label=lst:rst]
ceco.cec package
================

Conference of Evolutional Computation (CEC) benchmark functions.
----------------------------------------------------------------

ceco.cec.ackley module
----------------------
.. automodule:: ceco.cec.ackley
   :members:
   :undoc-members:
   :show-inheritance:

ceco.cec.bent_cigar module
--------------------------
.. automodule:: ceco.cec.bent_cigar
   :members:
   :undoc-members:
   :show-inheritance:
\end{lstlisting}
Pemisahan antara dokumentasi teknis internal berbasis docstring dan dokumentasi umum berbasis\ .rst menciptakan sistem dokumentasi yang komprehensif dan terstruktur. Pendekatan ini memungkinkan \textit{library} CECO menyediakan informasi yang mudah diakses baik untuk pengguna pemula maupun pengembang berpengalaman, sehingga meningkatkan kemudahan adopsi dan mempercepat integrasi dalam berbagai proyek penelitian optimisasi.

\section{\textit{Benchmarking}}
Dalam proses benchmarking, digunakan sebanyak 22 fungsi yang mencakup berbagai karakteristik, termasuk fungsi unimodal dan multimodal. Fungsi-fungsi ini memiliki berbagai tingkat kompleksitas dengan dimensi yang bervariasi, mulai dari 1, 2, 5, 10, 20, 25, 50, 100, 200, hingga 500. Keberagaman dimensi ini memungkinkan evaluasi algoritma dalam berbagai skenario optimasi, baik yang sederhana maupun kompleks.

Untuk melakukan benchmarking, digunakan dua algoritma optimasi, yaitu \textit{Komodo Mlipir Algorithm} (KMA) \citep{Suyanto_2022} dan \textit{Elephant Herding Optimization} (EHO) \citep{Wang_2015}. KMA terinspirasi dari pergerakan komodo dalam berburu mangsanya, sedangkan EHO didasarkan pada perilaku sosial kawanan gajah dalam mencari sumber daya. Kedua algoritma ini dipilih untuk menguji efektivitas dalam menyelesaikan permasalahan optimasi pada berbagai jenis fungsi \textit{benchmark} yang telah ditentukan.

\subsection{Pengaturan Parameter}
Pada algoritma \textit{Komodo Mlipir Algorithm} (KMA), jumlah populasi awal yang digunakan adalah 5 individu, dengan batas minimum populasi tetap 5 dan batas maksimum 200, serta mengalami perubahan populasi secara bertahap sebesar 5 individu. Selain itu, algoritma ini memiliki parameter tingkat mlipir untuk komodo kecil sebesar 0.5, sementara porsi komodo besar juga ditetapkan pada nilai yang sama, yaitu 0.5.

Di sisi lain, algoritma \textit{Elephant Herding Optimization} (EHO) menggunakan populasi 50 individu, yang dikelompokkan ke dalam 5 suku. Parameter tambahan dalam algoritma ini meliputi alpha dengan nilai 0.1 dan beta sebesar 0.1, yang mengatur dinamika eksplorasi dan eksploitasi selama proses optimasi berlangsung.

Baik KMA maupun EHO melaksanakan proses pencarian solusi dengan batas maksimum sebanyak 100 iterasi. Dengan demikian, evaluasi performa kedua algoritma dilakukan secara adil dalam jumlah perulangan yang sama, sehingga hasil yang diperoleh dapat dibandingkan secara setara. Selain itu, untuk memastikan konsistensi dan reliabilitas hasil, setiap algoritma dijalankan sebanyak 30 kali pengujian terpisah. Pendekatan ini bertujuan untuk mengurangi pengaruh faktor kebetulan atau kondisi awal yang bersifat acak, serta memperoleh gambaran performa algoritma yang lebih akurat dan representatif. Detail lengkap mengenai konfigurasi parameter yang digunakan pada masing-masing algoritma dapat ditemukan dalam \cref{tab:pengaturan-parameter}

% Please add the following required packages to your document preamble:
% \usepackage{multirow[t]}
\begin{table}[h!]
\centering
\caption{pengaturan parameter}
\label{tab:pengaturan-parameter}
\begin{tabular}{|l|l|l|}
\hline
algoritma            & parameter                 & nilai \\ \hline
\multirow[t]{7}{*}{KMA} & populasi awal             & 5     \\ \cline{2-3} 
                     & populasi minimal          & 5     \\ \cline{2-3} 
                     & populasi maksimal         & 200   \\ \cline{2-3} 
                     & tingkat mlipir            & 0.5   \\ \cline{2-3} 
                     & porsi komodo jantan besar & 0.5   \\ \cline{2-3} 
                     & perubahan populasi        & 5     \\ \cline{2-3} 
                     & maksimum iterasi          & 100   \\ \hline
\multirow[t]{5}{*}{EHO} & populasi                  & 50    \\ \cline{2-3} 
                     & jumlah suku               & 5     \\ \cline{2-3} 
                     & alpha                     & 0.5   \\ \cline{2-3} 
                     & beta                      & 0.1   \\ \cline{2-3} 
                     & maksimum iterasi          & 100   \\ \hline
\end{tabular}
\end{table}

\subsection{Hasil \textit{Benchmark}}
Berdasarkan hasil pengujian, \textit{Komodo Mlipir Algorithm} (KMA) menunjukkan performa yang lebih unggul dalam hal nilai terbaik (\textit{best result}) dibandingkan \textit{Elephant Herding Optimization} (EHO). KMA berhasil meraih 163 hasil terbaik dari 220 percobaan, sedangkan EHO hanya mencatatkan 42 hasil terbaik dari 220 percobaan yaitu pada fungsi different power dimensi 2 dan 5 (\cref{tab:eho-result} dan \cref{fig:graph_different_power}), ellipsoid dimensi 2 (\cref{tab:eho-result} dan \cref{fig:graph_different_power}), expanded schaffer f6 dimensi 5, 10, 20, 25, 100, 500 (\cref{tab:eho-result} dan \cref{fig:graph_expanded_schaffer_f6}), griewank dimensi 2 (\cref{tab:eho-result} dan \cref{fig:graph_griewank}), happycat dimensi 1 dan 2 (\cref{tab:eho-result} dan \cref{fig:graph_happycat}), hgbat dimensi 1, 2, 50 (\cref{tab:eho-result} dan \cref{fig:graph_hgbat}), levy dimensi 1, 2, 5, 25 (\cref{tab:eho-result} dan \cref{fig:graph_levy}), rastrigin dimensi 2 (\cref{tab:eho-result} dan \cref{fig:graph_rastrigin}), schwefel 1.2 problem dimensi 2, 20, 25 (\cref{tab:eho-result} dan \cref{fig:graph_schwefel_1_2_problem}), schwefel 2.13 di semua dimensi (\cref{tab:eho-result} dan \cref{fig:graph_schwefel_2_13_problem}), shubert dimensi 1 dan 2 (\cref{tab:eho-result} dan \cref{fig:graph_shubert}), sphere dimensi 2 (\cref{tab:eho-result} dan \cref{fig:graph_sphere}), vincent dimensi 1, 2, 10, 100 (\cref{tab:eho-result} dan \cref{fig:graph_vincent}).

Namun demikian, apabila dilihat dari aspek hasil terburuk (\textit{worst result}), rata-rata (\textit{mean}), dan simpangan baku (\textit{standard deviation}), EHO menunjukkan performa yang lebih stabil dibandingkan KMA.\ Secara rinci, EHO mencatat hasil terburuk sebanyak 149 dari 220, rata-rata sebesar 108 dari 220, dan simpangan baku sebesar 144 dari 220. Sementara itu, KMA memperoleh hasil terburuk sebesar 60 dari 220 yaitu pada ackley dimensi 5 (\cref{tab:kma-result} dan \cref{fig:graph_ackley}), bent cigar dimensi 10, 20, 50, 100, 200 (\cref{tab:kma-result} dan \cref{fig:graph_bent_cigar}), different power pada semua dimensi kecuali dimensi 1 (\cref{tab:kma-result} dan \cref{fig:graph_different_power}), discus dimensi 5 dan 20 (\cref{tab:kma-result} dan \cref{fig:graph_discus}), ellipsoid dimensi 5, 100, 200 (\cref{tab:kma-result} dan \cref{fig:graph_ellipsoid}), elliptic dimensi 100, 200, 500 (\cref{tab:kma-result} dan \cref{fig:graph_elliptic}), griewank semua dimensi kecuali dimensi 1, 2, 5 (\cref{tab:kma-result} dan \cref{fig:graph_griewank}), happycat dimensi 5, 10, 25 (\cref{tab:kma-result} dan \cref{fig:graph_happycat}), hgbat dimensi 5, 10, 20 (\cref{tab:kma-result} dan \cref{fig:graph_hgbat}), katsuura dimensi 2, 10, 100, 200, 500 (\cref{tab:kma-result} dan \cref{fig:graph_katsuura}), rastrigin dimensi 500 (\cref{tab:kma-result} dan \cref{fig:graph_rastrigin}), sphere di semua dimensi kecuali dimensi 1, 2, 5, 10 (\cref{tab:kma-result} dan \cref{fig:graph_sphere}), vincent dimensi 25, 100, 200, 500 (\cref{tab:kma-result} dan \cref{fig:graph_vincent}), weierstrass dimensi 20, 25, 50, 200, 500 (\cref{tab:kma-result} dan \cref{fig:graph_weierstrass}), zakharov dimensi 25, 50, 100 (\cref{tab:kma-result} dan \cref{fig:graph_zakharov}), rata-rata sebesar 100 dari 220, dan simpangan baku sebesar 63 dari 220 yaitu pada ackley dimensi 5 (\cref{tab:kma-result} dan \cref{fig:graph_ackley}), bent cigar semua dimensi kecuali dimensi 1, 2, 5 (\cref{tab:kma-result} dan \cref{fig:graph_bent_cigar}), different power semua dimensi kecuali dimensi 1 (\cref{tab:kma-result} dan \cref{fig:graph_different_power}), discus dimensi 5, 20 (\cref{tab:kma-result} dan \cref{fig:graph_discus}), ellipsoid dimensi 5, 25, 100, 200 (\cref{tab:kma-result} dan \cref{fig:graph_ellipsoid}), elliptic dimensi 20, 100, 200, 500 (\cref{tab:kma-result} dan \cref{fig:graph_elliptic}), expanded schaffer f6 dimensi 10, 20, 50, 100 (\cref{tab:kma-result} dan \cref{fig:graph_expanded_schaffer_f6}), griewank semua dimensi kecuali dimensi 1, 2, 5 (\cref{tab:kma-result} dan \cref{fig:graph_griewank}), hgbat dimensi 10, 20 (\cref{tab:kma-result} dan \cref{fig:graph_hgbat}), katsuura dimensi 2, 10, 100, 200, 500 (\cref{tab:kma-result} dan \cref{fig:graph_katsuura}), rastrigin dimensi 500 (\cref{tab:kma-result} dan \cref{fig:graph_rastrigin}), schwefel 1.2 problem dimensi 100 dan 500 (\cref{tab:kma-result} dan \cref{fig:graph_schwefel_1_2_problem}), sphere semua dimensi kecuali dimensi 1, 2, 5 (\cref{tab:kma-result} dan \cref{fig:graph_sphere}), vincent dimensi 100, 200, 500 (\cref{tab:kma-result} dan \cref{fig:graph_vincent}), weierstrass dimensi 500 (\cref{tab:kma-result} dan \cref{fig:graph_weierstrass}), zakharov dimensi 25, 50, 100, 200 (\cref{tab:kma-result} dan \cref{fig:graph_zakharov}).

Selain itu, terdapat hasil seri yang menunjukkan nilai sama antara kedua algoritma, yaitu 14 dari 220 untuk nilai terbaik yaitu pada step function semua dimensi (\cref{tab:eho-result}, \cref{tab:kma-result} dan \cref{fig:graph_step_function}), expanded schaffer f6 dimensi 1 (\cref{tab:eho-result}, \cref{tab:kma-result} dan \cref{fig:graph_expanded_schaffer_f6}), griewank dimensi 1 (\cref{tab:eho-result}, \cref{tab:kma-result} dan \cref{fig:graph_griewank}), rastrigin dimensi 1 (\cref{tab:eho-result}, \cref{tab:kma-result} dan \cref{fig:graph_rastrigin}), rosenbrock dimensi 1 (\cref{tab:eho-result}, \cref{tab:kma-result} dan \cref{fig:graph_rosenbrock}), 11 dari 220 untuk hasil terburuk yaitu pada step function semua dimensi (\cref{tab:eho-result}, \cref{tab:kma-result} dan \cref{fig:graph_step_function}) dan rosenbrock dimensi 1 (\cref{tab:eho-result}, \cref{tab:kma-result} dan \cref{fig:graph_rosenbrock}), 11 dari 220 untuk rata-rata  yaitu pada step function semua dimensi (\cref{tab:eho-result}, \cref{tab:kma-result} dan \cref{fig:graph_step_function}) dan rosenbrock dimensi 1 (\cref{tab:eho-result}, \cref{tab:kma-result} dan \cref{fig:graph_rosenbrock}), dan 11 dari 220 untuk simpangan baku yaitu pada step function semua dimensi (\cref{tab:eho-result}, \cref{tab:kma-result} dan \cref{fig:graph_step_function}) dan rosenbrock dimensi 1 (\cref{tab:eho-result}, \cref{tab:kma-result} dan \cref{fig:graph_rosenbrock}).

Pada pengujian ini juga ditemukan hasil invalid yang diakibatkan oleh \textit{overflow} nilai float64 dan longdouble pada \textit{library} Numpy, yaitu 1 dari 220 untuk nilai terbaik pada fungsi shubert dimensi 500 (\cref{tab:eho-result}, \cref{tab:kma-result} dan \cref{fig:graph_shubert}), 0 dari 220 untuk hasil terburuk, 1 dari 220 untuk rata-rata pada fungsi shubert dimensi 500 (\cref{tab:eho-result}, \cref{tab:kma-result} dan \cref{fig:graph_shubert}), dan 2 dari 220 untuk simpangan bakupada fungsi shubert dimensi 200 dan 500 (\cref{tab:eho-result}, \cref{tab:kma-result} dan \cref{fig:graph_shubert}). Nilai-nilai ini perlu diperhatikan karena dapat memengaruhi validitas evaluasi. Hasil lengkap \textit{benchmarking} pada dilihat pada \cref{tab:eho-result}, \cref{tab:kma-result}.

Dengan demikian, dapat disimpulkan bahwa KMA lebih unggul dalam mencapai solusi optimal, sedangkan EHO memiliki kelebihan dalam hal kestabilan dan konsistensi performa di berbagai kondisi pengujian. Temuan ini menunjukkan bahwa meskipun KMA lebih sering mencapai hasil terbaik, EHO cenderung menghasilkan nilai yang lebih konsisten dan stabil, meskipun tidak selalu mencapai performa optimal.

\begin{longtable}[c]{|p{3.5cm}|l|l|l|l|l|}
\caption{hasil \textit{benchmark} EHO}
\label{tab:eho-result}\\
\hline
function                                & dimension & best       & worst      & avg        & std       \\ \hline
\endfirsthead
%
\endhead
%
\multirow[t]{10}{*}{ackley}                & 1         & 1,92E-13   & 4,57E-08   & 1,69E-09   & 8,19E-09  \\ \cline{2-6} 
                                        & 2         & 6,48E-09   & 8,12E-03   & 8,83E-04   & 1,91E-03  \\ \cline{2-6} 
                                        & 5         & 2,61E-03   & 1,09E-01   & 1,95E-02   & 2,07E-02  \\ \cline{2-6} 
                                        & 10        & 4,70E-03   & 1,18E-01   & 3,46E-02   & 2,56E-02  \\ \cline{2-6} 
                                        & 20        & 3,53E-03   & 1,48E-01   & 4,46E-02   & 3,24E-02  \\ \cline{2-6} 
                                        & 25        & 4,54E-03   & 1,12E-01   & 3,99E-02   & 2,58E-02  \\ \cline{2-6} 
                                        & 50        & 1,35E-02   & 1,33E-01   & 5,15E-02   & 2,78E-02  \\ \cline{2-6} 
                                        & 100       & 8,41E-03   & 1,98E-01   & 5,11E-02   & 4,34E-02  \\ \cline{2-6} 
                                        & 200       & 6,86E-03   & 1,23E-01   & 5,39E-02   & 3,17E-02  \\ \cline{2-6} 
                                        & 500       & 7,14E-03   & 1,39E-01   & 5,61E-02   & 3,03E-02  \\ \hline
\multirow[t]{10}{*}{bent cigar}            & 1         & 2,61E-27   & 2,59E-18   & 8,92E-20   & 4,65E-19  \\ \cline{2-6} 
                                        & 2         & 9,96E-05   & 9,80E-01   & 1,41E-01   & 2,59E-01  \\ \cline{2-6} 
                                        & 5         & 2,04E-01   & 1,46E+02   & 2,06E+01   & 3,31E+01  \\ \cline{2-6} 
                                        & 10        & 8,20E+00   & 2,18E+03   & 3,43E+02   & 5,50E+02  \\ \cline{2-6} 
                                        & 20        & 1,06E+01   & 1,50E+04   & 2,44E+03   & 3,31E+03  \\ \cline{2-6} 
                                        & 25        & 5,96E+00   & 9,50E+03   & 1,93E+03   & 2,60E+03  \\ \cline{2-6} 
                                        & 50        & 2,33E+02   & 4,94E+04   & 1,02E+04   & 1,13E+04  \\ \cline{2-6} 
                                        & 100       & 2,27E+02   & 4,33E+04   & 9,69E+03   & 8,61E+03  \\ \cline{2-6} 
                                        & 200       & 1,04E+03   & 1,83E+05   & 4,62E+04   & 4,59E+04  \\ \cline{2-6} 
                                        & 500       & 2,74E+03   & 3,00E+05   & 9,45E+04   & 7,46E+04  \\ \hline
\multirow[t]{10}{*}{different power}       & 1         & 0.00E+00   & 1,87E-07   & 1,03E-08   & 3,92E-08  \\ \cline{2-6} 
                                        & 2         & 1,00E-10   & 5,89E-03   & 2,92E-04   & 1,05E-03  \\ \cline{2-6} 
                                        & 5         & 1,84E-06   & 1,27E-02   & 1,99E-03   & 2,61E-03  \\ \cline{2-6} 
                                        & 10        & 8,11E-05   & 2,38E-02   & 3,25E-03   & 4,49E-03  \\ \cline{2-6} 
                                        & 20        & 5,98E-04   & 1,77E-02   & 4,40E-03   & 3,79E-03  \\ \cline{2-6} 
                                        & 25        & 1,83E-04   & 3,26E-02   & 7,43E-03   & 7,46E-03  \\ \cline{2-6} 
                                        & 50        & 1,56E-03   & 4,03E-02   & 1,44E-02   & 1,26E-02  \\ \cline{2-6} 
                                        & 100       & 3,27E-03   & 6,03E-02   & 2,74E-02   & 1,68E-02  \\ \cline{2-6} 
                                        & 200       & 5,06E-03   & 9,12E-02   & 4,03E-02   & 2,32E-02  \\ \cline{2-6} 
                                        & 500       & 6,21E-03   & 1,86E-01   & 7,29E-02   & 5,18E-02  \\ \hline
\multirow[t]{10}{*}{discus}                & 1         & 0.00E+00   & 1,02E-09   & 0.00E+00   & 1,82E-10  \\ \cline{2-6} 
                                        & 2         & 1,05E-03   & 1,14E+00   & 1,98E-01   & 2,71E-01  \\ \cline{2-6} 
                                        & 5         & 2,97E-03   & 2,38E+00   & 3,68E-01   & 5,73E-01  \\ \cline{2-6} 
                                        & 10        & 2,42E-03   & 2,91E+00   & 5,56E-01   & 7,28E-01  \\ \cline{2-6} 
                                        & 20        & 1,24E-02   & 7,03E+00   & 8,47E-01   & 1,37E+00  \\ \cline{2-6} 
                                        & 25        & 2,76E-03   & 1,60E+00   & 4,16E-01   & 4,08E-01  \\ \cline{2-6} 
                                        & 50        & 4,86E-03   & 7,15E+00   & 7,68E-01   & 1,35E+00  \\ \cline{2-6} 
                                        & 100       & 3,57E-02   & 1,80E+00   & 6,83E-01   & 5,34E-01  \\ \cline{2-6} 
                                        & 200       & 4,05E-02   & 4,59E+00   & 6,60E-01   & 9,16E-01  \\ \cline{2-6} 
                                        & 500       & 1,19E-02   & 3,10E+00   & 8,40E-01   & 7,91E-01  \\ \hline
\multirow[t]{10}{*}{ellipsoid}             & 1         & 1,17E-27   & 1,37E-18   & 6,03E-20   & 2,53E-19  \\ \cline{2-6} 
                                        & 2         & 0.00E+00   & 1,05E-06   & 8,22E-08   & 2,56E-07  \\ \cline{2-6} 
                                        & 5         & 1,93E-07   & 1,85E-03   & 1,68E-04   & 3,31E-04  \\ \cline{2-6} 
                                        & 10        & 3,90E-05   & 3,34E-02   & 2,82E-03   & 6,14E-03  \\ \cline{2-6} 
                                        & 20        & 3,48E-04   & 1,33E-01   & 2,60E-02   & 3,34E-02  \\ \cline{2-6} 
                                        & 25        & 3,62E-04   & 1,36E-01   & 3,42E-02   & 3,26E-02  \\ \cline{2-6} 
                                        & 50        & 4,38E-03   & 6,84E-01   & 1,48E-01   & 1,61E-01  \\ \cline{2-6} 
                                        & 100       & 4,45E-02   & 2,63E+00   & 8,39E-01   & 6,82E-01  \\ \cline{2-6} 
                                        & 200       & 4,15E-01   & 9,33E+00   & 3,36E+00   & 2,38E+00  \\ \cline{2-6} 
                                        & 500       & 7,06E-01   & 1,12E+02   & 3,19E+01   & 2,52E+01  \\ \hline
\multirow[t]{10}{*}{elliptic}              & 1         & 6,46E-14   & 1,05E-02   & 3,68E-04   & 1,89E-03  \\ \cline{2-6} 
                                        & 2         & 2,53E-04   & 9,73E-01   & 1,22E-01   & 2,45E-01  \\ \cline{2-6} 
                                        & 5         & 3,20E-03   & 8,47E+00   & 1,13E+00   & 1,80E+00  \\ \cline{2-6} 
                                        & 10        & 2,21E-02   & 2,19E+02   & 1,03E+01   & 3,89E+01  \\ \cline{2-6} 
                                        & 20        & 3,30E-01   & 3,62E+02   & 2,85E+01   & 6,99E+01  \\ \cline{2-6} 
                                        & 25        & 1,89E-01   & 1,37E+02   & 2,91E+01   & 4,08E+01  \\ \cline{2-6} 
                                        & 50        & 2,37E+00   & 5,80E+02   & 1,19E+02   & 1,52E+02  \\ \cline{2-6} 
                                        & 100       & 6,34E+00   & 3,79E+03   & 6,71E+02   & 8,68E+02  \\ \cline{2-6} 
                                        & 200       & 2,19E+01   & 5,92E+03   & 1,66E+03   & 1,74E+03  \\ \cline{2-6} 
                                        & 500       & 2,46E+02   & 1,96E+04   & 6,67E+03   & 5,61E+03  \\ \hline
\multirow[t]{10}{*}{expanded schaffer f6}  & 1         & 0.00E+00   & 1,11E-16   & 3,70E-18   & 1,99E-17  \\ \cline{2-6} 
                                        & 2         & 0.00E+00   & 1,94E-02   & 1,49E-02   & 8,22E-03  \\ \cline{2-6} 
                                        & 5         & 2,25E-05   & 1,38E+00   & 4,79E-01   & 3,12E-01  \\ \cline{2-6} 
                                        & 10        & 2,53E-05   & 3,72E+00   & 1,96E+00   & 1,02E+00  \\ \cline{2-6} 
                                        & 20        & 1,90E-03   & 7,57E+00   & 5,72E+00   & 2,04E+00  \\ \cline{2-6} 
                                        & 25        & 4,15E-03   & 1,08E+01   & 7,90E+00   & 2,36E+00  \\ \cline{2-6} 
                                        & 50        & 5,51E-03   & 2,19E+01   & 1,61E+01   & 7,35E+00  \\ \cline{2-6} 
                                        & 100       & 1,97E-03   & 4,50E+01   & 3,23E+01   & 1,79E+01  \\ \cline{2-6} 
                                        & 200       & 8,06E+01   & 9,47E+01   & 8,91E+01   & 3,42E+00  \\ \cline{2-6} 
                                        & 500       & 1,74E+00   & 2,41E+02   & 2,25E+02   & 4,18E+01  \\ \hline
\multirow[t]{10}{*}{griewank}              & 1         & 0.00E+00   & 0.00E+00   & 0.00E+00   & 0.00E+00  \\ \cline{2-6} 
                                        & 2         & 0.00E+00   & 4,86E-02   & 8,69E-03   & 1,29E-02  \\ \cline{2-6} 
                                        & 5         & 3,73E-07   & 1,97E-01   & 6,61E-03   & 3,54E-02  \\ \cline{2-6} 
                                        & 10        & 1,97E-07   & 1,21E-03   & 1,17E-04   & 2,87E-04  \\ \cline{2-6} 
                                        & 20        & 6,68E-06   & 5,41E-04   & 9,75E-05   & 1,26E-04  \\ \cline{2-6} 
                                        & 25        & 6,19E-06   & 1,02E-03   & 1,84E-04   & 2,32E-04  \\ \cline{2-6} 
                                        & 50        & 2,14E-06   & 8,29E-04   & 1,66E-04   & 2,01E-04  \\ \cline{2-6} 
                                        & 100       & 5,52E-06   & 3,69E-03   & 4,51E-04   & 6,62E-04  \\ \cline{2-6} 
                                        & 200       & 2,35E-06   & 1,21E-03   & 2,73E-04   & 2,99E-04  \\ \cline{2-6} 
                                        & 500       & 1,02E-05   & 1,40E-03   & 6,23E-04   & 4,87E-04  \\ \hline
\multirow[t]{10}{*}{happycat}              & 1         & 1,01E-03   & 1,30E-02   & 3,18E-03   & 2,99E-03  \\ \cline{2-6} 
                                        & 2         & 2,67E-03   & 2,68E-01   & 6,41E-02   & 6,44E-02  \\ \cline{2-6} 
                                        & 5         & 1,55E-01   & 1,25E+00   & 5,83E-01   & 2,61E-01  \\ \cline{2-6} 
                                        & 10        & 3,63E-01   & 1,42E+00   & 8,01E-01   & 2,64E-01  \\ \cline{2-6} 
                                        & 20        & 6,84E-01   & 1,31E+00   & 1,00E+00   & 2,07E-01  \\ \cline{2-6} 
                                        & 25        & 7,46E-01   & 1,47E+00   & 1,06E+00   & 1,98E-01  \\ \cline{2-6} 
                                        & 50        & 7,89E-01   & 1,25E+00   & 1,10E+00   & 1,03E-01  \\ \cline{2-6} 
                                        & 100       & 9,44E-01   & 1,40E+00   & 1,13E+00   & 1,11E-01  \\ \cline{2-6} 
                                        & 200       & 9,34E-01   & 1,67E+00   & 1,14E+00   & 1,65E-01  \\ \cline{2-6} 
                                        & 500       & 9,90E-01   & 1,77E+00   & 1,23E+00   & 1,97E-01  \\ \hline
\multirow[t]{10}{*}{hgbat}                 & 1         & 4,92E-06   & 7,97E-02   & 6,74E-03   & 1,76E-02  \\ \cline{2-6} 
                                        & 2         & 7,11E-04   & 4,90E-01   & 1,23E-01   & 9,86E-02  \\ \cline{2-6} 
                                        & 5         & 2,47E-01   & 5,56E-01   & 4,46E-01   & 7,48E-02  \\ \cline{2-6} 
                                        & 10        & 4,01E-01   & 7,48E-01   & 5,27E-01   & 7,62E-02  \\ \cline{2-6} 
                                        & 20        & 4,65E-01   & 9,14E-01   & 5,58E-01   & 8,64E-02  \\ \cline{2-6} 
                                        & 25        & 4,22E-01   & 7,61E-01   & 5,59E-01   & 6,92E-02  \\ \cline{2-6} 
                                        & 50        & 4,74E-01   & 8,03E-01   & 5,81E-01   & 8,71E-02  \\ \cline{2-6} 
                                        & 100       & 5,06E-01   & 1,03E+00   & 5,69E-01   & 9,93E-02  \\ \cline{2-6} 
                                        & 200       & 5,03E-01   & 6,30E-01   & 5,40E-01   & 2,71E-02  \\ \cline{2-6} 
                                        & 500       & 5,16E-01   & 1,30E+00   & 5,81E-01   & 1,39E-01  \\ \hline
\multirow[t]{10}{*}{katsuura}              & 1         & 2,96E-09   & 3,50E-01   & 2,52E-02   & 7,37E-02  \\ \cline{2-6} 
                                        & 2         & 4,93E-05   & 1,42E+00   & 2,92E-01   & 3,65E-01  \\ \cline{2-6} 
                                        & 5         & 2,51E-01   & 1,39E+00   & 7,17E-01   & 2,77E-01  \\ \cline{2-6} 
                                        & 10        & 6,77E-01   & 2,97E+00   & 1,49E+00   & 5,42E-01  \\ \cline{2-6} 
                                        & 20        & 9,68E-01   & 3,98E+00   & 2,42E+00   & 6,90E-01  \\ \cline{2-6} 
                                        & 25        & 1,93E+00   & 3,75E+00   & 2,84E+00   & 4,76E-01  \\ \cline{2-6} 
                                        & 50        & 2,57E+00   & 5,82E+00   & 4,28E+00   & 7,92E-01  \\ \cline{2-6} 
                                        & 100       & 3,85E+00   & 6,28E+00   & 4,82E+00   & 6,27E-01  \\ \cline{2-6} 
                                        & 200       & 3,14E+00   & 4,39E+00   & 3,77E+00   & 3,33E-01  \\ \cline{2-6} 
                                        & 500       & 1,45E+00   & 1,77E+00   & 1,62E+00   & 7,80E-02  \\ \hline
\multirow[t]{10}{*}{levy}                  & 1         & 3,06E-26   & 3,36E-20   & 2,10E-21   & 7,55E-21  \\ \cline{2-6} 
                                        & 2         & 8,81E-20   & 2,21E-04   & 1,83E-05   & 5,16E-05  \\ \cline{2-6} 
                                        & 5         & 4,24E-04   & 5,41E-02   & 7,87E-03   & 1,05E-02  \\ \cline{2-6} 
                                        & 10        & 5,36E-03   & 1,49E-01   & 3,49E-02   & 3,87E-02  \\ \cline{2-6} 
                                        & 20        & 1,13E-02   & 4,90E-01   & 6,15E-02   & 1,00E-01  \\ \cline{2-6} 
                                        & 25        & 1,06E-02   & 1,77E-01   & 5,92E-02   & 4,36E-02  \\ \cline{2-6} 
                                        & 50        & 2,30E-02   & 4,93E-01   & 1,16E-01   & 1,06E-01  \\ \cline{2-6} 
                                        & 100       & 4,75E-02   & 5,38E-01   & 1,87E-01   & 1,33E-01  \\ \cline{2-6} 
                                        & 200       & 9,44E-02   & 2,59E+00   & 4,72E-01   & 5,19E-01  \\ \cline{2-6} 
                                        & 500       & 2,80E-01   & 3,83E+00   & 1,03E+00   & 8,28E-01  \\ \hline
\multirow[t]{10}{*}{rastrigin}             & 1         & 0.00E+00   & 2,76E-12   & 9,52E-14   & 4,96E-13  \\ \cline{2-6} 
                                        & 2         & 2,53E-10   & 1,20E+00   & 4,35E-02   & 2,14E-01  \\ \cline{2-6} 
                                        & 5         & 5,94E-04   & 1,73E+00   & 1,15E-01   & 3,18E-01  \\ \cline{2-6} 
                                        & 10        & 1,40E-03   & 1,77E+00   & 2,71E-01   & 3,70E-01  \\ \cline{2-6} 
                                        & 20        & 6,06E-03   & 3,58E+00   & 8,32E-01   & 9,03E-01  \\ \cline{2-6} 
                                        & 25        & 1,84E-02   & 3,37E+00   & 9,56E-01   & 8,46E-01  \\ \cline{2-6} 
                                        & 50        & 3,22E-01   & 1,66E+01   & 2,90E+00   & 2,91E+00  \\ \cline{2-6} 
                                        & 100       & 1,17E-01   & 1,31E+01   & 3,91E+00   & 3,17E+00  \\ \cline{2-6} 
                                        & 200       & 4,27E-01   & 3,46E+01   & 8,91E+00   & 8,13E+00  \\ \cline{2-6} 
                                        & 500       & 5,32E-01   & 1,12E+02   & 2,47E+01   & 2,37E+01  \\ \hline
\multirow[t]{10}{*}{rosenbrock}            & 1         & 0.00E+00   & 0.00E+00   & 0.00E+00   & 0.00E+00  \\ \cline{2-6} 
                                        & 2         & 5,12E-08   & 2,72E+01   & 2,52E+00   & 6,71E+00  \\ \cline{2-6} 
                                        & 5         & 5,29E-02   & 3,11E+00   & 6,11E-01   & 7,42E-01  \\ \cline{2-6} 
                                        & 10        & 2,24E-01   & 2,79E+00   & 1,20E+00   & 6,41E-01  \\ \cline{2-6} 
                                        & 20        & 4,94E-01   & 6,38E+00   & 2,69E+00   & 1,52E+00  \\ \cline{2-6} 
                                        & 25        & 4,56E-01   & 1,24E+01   & 3,74E+00   & 2,41E+00  \\ \cline{2-6} 
                                        & 50        & 1,25E+00   & 2,76E+01   & 9,77E+00   & 5,83E+00  \\ \cline{2-6} 
                                        & 100       & 1,01E+00   & 3,66E+01   & 1,64E+01   & 7,07E+00  \\ \cline{2-6} 
                                        & 200       & 4,22E+00   & 5,75E+01   & 3,01E+01   & 1,13E+01  \\ \cline{2-6} 
                                        & 500       & 1,16E+01   & 2,03E+02   & 7,63E+01   & 4,23E+01  \\ \hline
\multirow[t]{10}{*}{schwefel1.2 problem}   & 1         & 1,19E-26   & 1,32E-18   & 4,41E-20   & 2,37E-19  \\ \cline{2-6} 
                                        & 2         & 1,11E-12   & 9,16E-06   & 7,44E-07   & 1,99E-06  \\ \cline{2-6} 
                                        & 5         & 2,44E-06   & 2,01E-03   & 3,18E-04   & 4,77E-04  \\ \cline{2-6} 
                                        & 10        & 1,28E-05   & 5,92E-03   & 9,69E-04   & 1,22E-03  \\ \cline{2-6} 
                                        & 20        & 1,72E-04   & 7,19E-01   & 3,38E-02   & 1,28E-01  \\ \cline{2-6} 
                                        & 25        & 4,29E-04   & 1,73E-01   & 2,10E-02   & 3,49E-02  \\ \cline{2-6} 
                                        & 50        & 4,33E-04   & 9,66E-01   & 1,46E-01   & 2,10E-01  \\ \cline{2-6} 
                                        & 100       & 4,96E-02   & 3,75E+00   & 9,42E-01   & 9,98E-01  \\ \cline{2-6} 
                                        & 200       & 2,56E-01   & 3,12E+01   & 6,07E+00   & 6,31E+00  \\ \cline{2-6} 
                                        & 500       & 1,72E+00   & 4,76E+02   & 9,37E+01   & 1,23E+02  \\ \hline
\multirow[t]{10}{*}{schwefel 2.13 problem} & 1         & 1,84E-21   & 1,05E-02   & 3,49E-04   & 1,88E-03  \\ \cline{2-6} 
                                        & 2         & 1,02E-11   & 8,00E+00   & 7,57E-01   & 1,87E+00  \\ \cline{2-6} 
                                        & 5         & 1,23E+02   & 6,26E+03   & 2,63E+03   & 1,59E+03  \\ \cline{2-6} 
                                        & 10        & 7,28E+03   & 1,03E+05   & 5,20E+04   & 2,47E+04  \\ \cline{2-6} 
                                        & 20        & 3,30E+05   & 1,01E+06   & 6,84E+05   & 1,58E+05  \\ \cline{2-6} 
                                        & 25        & 5,79E+05   & 1,49E+06   & 1,07E+06   & 2,44E+05  \\ \cline{2-6} 
                                        & 50        & 4,53E+06   & 8,41E+06   & 6,67E+06   & 9,21E+05  \\ \cline{2-6} 
                                        & 100       & 2,79E+07   & 4,20E+07   & 3,57E+07   & 3,57E+06  \\ \cline{2-6} 
                                        & 200       & 1,55E+08   & 1,97E+08   & 1,75E+08   & 9,74E+06  \\ \cline{2-6} 
                                        & 500       & 1,16E+09   & 1,40E+09   & 1,32E+09   & 4,82E+07  \\ \hline
\multirow[t]{10}{*}{shubert}               & 1         & -1,45E+01  & -1,45E+01  & -1,45E+01  & 4,79E-03  \\ \cline{2-6} 
                                        & 2         & -2,10E+02  & -1,45E+02  & -2,04E+02  & 1,39E+01  \\ \cline{2-6} 
                                        & 5         & -3,12E+05  & -2,24E+04  & -1,00E+05  & 7,84E+04  \\ \cline{2-6} 
                                        & 10        & -1,30E+10  & -2,81E+07  & -8,91E+08  & 2,33E+09  \\ \cline{2-6} 
                                        & 20        & -1,09E+17  & -1,08E+13  & -5,18E+15  & 1,97E+16  \\ \cline{2-6} 
                                        & 25        & -3,65E+19  & -7,28E+15  & -3,15E+18  & 7,71E+18  \\ \cline{2-6} 
                                        & 50        & -3,99E+35  & -6,51E+28  & -1,39E+34  & 7,16E+34  \\ \cline{2-6} 
                                        & 100       & -2,42E+66  & -6,29E+51  & -8,08E+64  & 4,35E+65  \\ \cline{2-6} 
                                        & 200       & -7,43E+116 & -4,80E+97  & -2,48E+115 & 1,33E+116 \\ \cline{2-6} 
                                        & 500       & -4,09E+265 & -4,54E+228 & -1,36E+264 & INF       \\ \hline
\multirow[t]{10}{*}{sphere}                & 1         & 3,21E-26   & 6,19E-16   & 2,37E-17   & 1,12E-16  \\ \cline{2-6} 
                                        & 2         & 6,31E-19   & 1,69E-06   & 1,27E-07   & 3,82E-07  \\ \cline{2-6} 
                                        & 5         & 4,46E-07   & 1,41E-03   & 1,03E-04   & 2,60E-04  \\ \cline{2-6} 
                                        & 10        & 1,73E-06   & 3,76E-03   & 4,73E-04   & 7,73E-04  \\ \cline{2-6} 
                                        & 20        & 2,01E-05   & 9,56E-03   & 1,72E-03   & 2,28E-03  \\ \cline{2-6} 
                                        & 25        & 2,54E-05   & 2,09E-02   & 3,94E-03   & 5,45E-03  \\ \cline{2-6} 
                                        & 50        & 4,29E-04   & 3,65E-02   & 7,10E-03   & 8,51E-03  \\ \cline{2-6} 
                                        & 100       & 3,09E-03   & 5,82E-02   & 1,71E-02   & 1,37E-02  \\ \cline{2-6} 
                                        & 200       & 1,44E-03   & 1,05E-01   & 3,08E-02   & 2,63E-02  \\ \cline{2-6} 
                                        & 500       & 1,12E-03   & 3,72E-01   & 9,30E-02   & 8,75E-02  \\ \hline
\multirow[t]{10}{*}{step function}         & 1         & 0.00E+00   & 0.00E+00   & 0.00E+00   & 0.00E+00  \\ \cline{2-6} 
                                        & 2         & 0.00E+00   & 0.00E+00   & 0.00E+00   & 0.00E+00  \\ \cline{2-6} 
                                        & 5         & 0.00E+00   & 0.00E+00   & 0.00E+00   & 0.00E+00  \\ \cline{2-6} 
                                        & 10        & 0.00E+00   & 0.00E+00   & 0.00E+00   & 0.00E+00  \\ \cline{2-6} 
                                        & 20        & 0.00E+00   & 0.00E+00   & 0.00E+00   & 0.00E+00  \\ \cline{2-6} 
                                        & 25        & 0.00E+00   & 0.00E+00   & 0.00E+00   & 0.00E+00  \\ \cline{2-6} 
                                        & 50        & 0.00E+00   & 0.00E+00   & 0.00E+00   & 0.00E+00  \\ \cline{2-6} 
                                        & 100       & 0.00E+00   & 0.00E+00   & 0.00E+00   & 0.00E+00  \\ \cline{2-6} 
                                        & 200       & 0.00E+00   & 0.00E+00   & 0.00E+00   & 0.00E+00  \\ \cline{2-6} 
                                        & 500       & 0.00E+00   & 0.00E+00   & 0.00E+00   & 0.00E+00  \\ \hline
\multirow[t]{10}{*}{vincent}               & 1         & -1,00E+00  & -1,00E+00  & -1,00E+00  & 0.00E+00  \\ \cline{2-6} 
                                        & 2         & -1,00E+00  & -9,96E-01  & -1,00E+00  & 1,02E-03  \\ \cline{2-6} 
                                        & 5         & -9,93E-01  & -8,41E-01  & -9,43E-01  & 4,32E-02  \\ \cline{2-6} 
                                        & 10        & -9,99E-01  & -6,41E-01  & -8,05E-01  & 1,02E-01  \\ \cline{2-6} 
                                        & 20        & -9,80E-01  & -3,99E-01  & -5,48E-01  & 1,04E-01  \\ \cline{2-6} 
                                        & 25        & -6,79E-01  & -3,58E-01  & -5,03E-01  & 7,97E-02  \\ \cline{2-6} 
                                        & 50        & -9,51E-01  & -2,44E-01  & -3,92E-01  & 1,19E-01  \\ \cline{2-6} 
                                        & 100       & -9,99E-01  & -1,99E-01  & -4,07E-01  & 2,62E-01  \\ \cline{2-6} 
                                        & 200       & -9,98E-01  & -1,59E-01  & -3,25E-01  & 2,62E-01  \\ \cline{2-6} 
                                        & 500       & -9,99E-01  & -1,44E-01  & -6,57E-01  & 3,90E-01  \\ \hline
\multirow[t]{10}{*}{weierstrass}           & 1         & 2,56E-11   & 4,37E-02   & 1,68E-03   & 7,85E-03  \\ \cline{2-6} 
                                        & 2         & 6,51E-05   & 6,70E-01   & 1,20E-01   & 1,70E-01  \\ \cline{2-6} 
                                        & 5         & 1,55E+00   & 4,19E+00   & 2,77E+00   & 6,53E-01  \\ \cline{2-6} 
                                        & 10        & 6,75E+00   & 1,22E+01   & 9,87E+00   & 1,29E+00  \\ \cline{2-6} 
                                        & 20        & 2,18E+01   & 2,82E+01   & 2,58E+01   & 1,67E+00  \\ \cline{2-6} 
                                        & 25        & 2,73E+01   & 3,74E+01   & 3,31E+01   & 2,23E+00  \\ \cline{2-6} 
                                        & 50        & 6,67E+01   & 8,15E+01   & 7,62E+01   & 3,38E+00  \\ \cline{2-6} 
                                        & 100       & 1,59E+02   & 1,72E+02   & 1,66E+02   & 4,41E+00  \\ \cline{2-6} 
                                        & 200       & 3,44E+02   & 3,61E+02   & 3,52E+02   & 4,31E+00  \\ \cline{2-6} 
                                        & 500       & 9,01E+02   & 9,42E+02   & 9,25E+02   & 9,11E+00  \\ \hline
\multirow[t]{10}{*}{zakharov}              & 1         & 8,72E-26   & 9,54E-15   & 3,23E-16   & 1,71E-15  \\ \cline{2-6} 
                                        & 2         & 4,43E-14   & 2,91E-06   & 2,58E-07   & 7,07E-07  \\ \cline{2-6} 
                                        & 5         & 5,89E-06   & 1,22E-03   & 1,70E-04   & 2,52E-04  \\ \cline{2-6} 
                                        & 10        & 1,90E-05   & 2,36E-03   & 6,76E-04   & 6,40E-04  \\ \cline{2-6} 
                                        & 20        & 3,68E-05   & 1,48E-02   & 3,18E-03   & 3,82E-03  \\ \cline{2-6} 
                                        & 25        & 1,62E-04   & 6,50E-02   & 8,92E-03   & 1,44E-02  \\ \cline{2-6} 
                                        & 50        & 1,15E-04   & 1,45E-01   & 3,38E-02   & 3,77E-02  \\ \cline{2-6} 
                                        & 100       & 2,05E-03   & 9,24E-01   & 9,26E-02   & 1,66E-01  \\ \cline{2-6} 
                                        & 200       & 2,70E-03   & 1,62E+00   & 3,18E-01   & 3,96E-01  \\ \cline{2-6} 
                                        & 500       & 8,30E-02   & 9,87E+00   & 1,66E+00   & 2,02E+00  \\ \hline
\end{longtable}

% |p{3cm}|l|p{3cm}|p{3cm}|p{3cm}|p{3cm}|
\begin{longtable}[c]{|p{3.5cm}|l|l|l|l|l|}
\caption{hasil \textit{benchmark} KMA}
\label{tab:kma-result}\\
\hline
function                                & dimension & best       & worst      & avg        & std       \\ \hline
\endfirsthead
% 
\endhead
% 
\multirow[t]{10}{*}{ackley}                & 1         & 4,44E-16   & 6,55E-04   & 7,76E-05   & 1,51E-04  \\ \cline{2-6} 
                                        & 2         & 4,44E-16   & 5,61E-02   & 9,25E-03   & 1,52E-02  \\ \cline{2-6} 
                                        & 5         & 2,50E-05   & 2,41E-02   & 4,87E-03   & 5,52E-03  \\ \cline{2-6} 
                                        & 10        & 2,00E-06   & 1,42E+00   & 6,95E-02   & 2,64E-01  \\ \cline{2-6} 
                                        & 20        & 1,04E-05   & 1,80E+01   & 1,65E+00   & 4,40E+00  \\ \cline{2-6} 
                                        & 25        & 1,52E-05   & 2,04E+00   & 1,29E-01   & 3,89E-01  \\ \cline{2-6} 
                                        & 50        & 4,59E-05   & 2,00E+01   & 1,46E+00   & 4,99E+00  \\ \cline{2-6} 
                                        & 100       & 1,58E-05   & 2,00E+01   & 2,87E+00   & 6,75E+00  \\ \cline{2-6} 
                                        & 200       & 1,90E-05   & 1,99E+01   & 1,41E+00   & 4,21E+00  \\ \cline{2-6} 
                                        & 500       & 6,49E-05   & 2,00E+01   & 8,25E-01   & 3,59E+00  \\ \hline
\multirow[t]{10}{*}{bent cigar}            & 1         & 0.00E+00   & 2,23E-07   & 9,57E-09   & 4,10E-08  \\ \cline{2-6} 
                                        & 2         & 7,72E-07   & 5,23E+00   & 5,80E-01   & 1,30E+00  \\ \cline{2-6} 
                                        & 5         & 1,09E-04   & 3,78E+02   & 1,86E+01   & 6,79E+01  \\ \cline{2-6} 
                                        & 10        & 1,33E-03   & 1,39E+03   & 1,13E+02   & 3,11E+02  \\ \cline{2-6} 
                                        & 20        & 8,80E-03   & 2,93E+03   & 1,45E+02   & 5,43E+02  \\ \cline{2-6} 
                                        & 25        & 6,13E-04   & 1,17E+04   & 5,79E+02   & 2,13E+03  \\ \cline{2-6} 
                                        & 50        & 1,05E-04   & 2,96E+04   & 1,38E+03   & 5,48E+03  \\ \cline{2-6} 
                                        & 100       & 5,59E-04   & 1,74E+04   & 1,13E+03   & 3,27E+03  \\ \cline{2-6} 
                                        & 200       & 3,86E-02   & 4,95E+03   & 4,72E+02   & 1,02E+03  \\ \cline{2-6} 
                                        & 500       & 1,71E-03   & 3,86E+05   & 1,36E+04   & 6,91E+04  \\ \hline
\multirow[t]{10}{*}{different power}       & 1         & 0.00E+00   & 4,87E-04   & 3,27E-05   & 9,18E-05  \\ \cline{2-6} 
                                        & 2         & 4,98E-08   & 5,64E-04   & 7,94E-05   & 1,50E-04  \\ \cline{2-6} 
                                        & 5         & 3,91E-06   & 5,75E-03   & 6,58E-04   & 1,33E-03  \\ \cline{2-6} 
                                        & 10        & 1,74E-06   & 6,95E-03   & 1,16E-03   & 1,93E-03  \\ \cline{2-6} 
                                        & 20        & 1,17E-06   & 6,43E-03   & 8,08E-04   & 1,32E-03  \\ \cline{2-6} 
                                        & 25        & 4,19E-06   & 6,40E-03   & 1,43E-03   & 1,83E-03  \\ \cline{2-6} 
                                        & 50        & 5,36E-07   & 1,96E-02   & 2,39E-03   & 4,38E-03  \\ \cline{2-6} 
                                        & 100       & 1,13E-05   & 2,33E-02   & 2,61E-03   & 4,68E-03  \\ \cline{2-6} 
                                        & 200       & 2,24E-06   & 1,83E-02   & 2,70E-03   & 4,25E-03  \\ \cline{2-6} 
                                        & 500       & 4,30E-06   & 6,61E-02   & 6,03E-03   & 1,29E-02  \\ \hline
\multirow[t]{10}{*}{discus}                & 1         & 0.00E+00   & 1,70E-01   & 7,25E-03   & 3,07E-02  \\ \cline{2-6} 
                                        & 2         & 6,93E-10   & 7,74E+00   & 5,38E-01   & 1,49E+00  \\ \cline{2-6} 
                                        & 5         & 5,87E-07   & 9,69E-01   & 1,78E-01   & 3,06E-01  \\ \cline{2-6} 
                                        & 10        & 7,08E-06   & 3,66E+00   & 3,50E-01   & 9,01E-01  \\ \cline{2-6} 
                                        & 20        & 4,04E-07   & 2,25E+00   & 1,23E-01   & 4,38E-01  \\ \cline{2-6} 
                                        & 25        & 1,55E-05   & 7,12E+00   & 6,33E-01   & 1,48E+00  \\ \cline{2-6} 
                                        & 50        & 2,37E-07   & 4,46E+01   & 2,55E+00   & 8,18E+00  \\ \cline{2-6} 
                                        & 100       & 2,30E-08   & 7,93E+01   & 3,91E+00   & 1,46E+01  \\ \cline{2-6} 
                                        & 200       & 3,87E-07   & 8,08E+01   & 3,10E+00   & 1,45E+01  \\ \cline{2-6} 
                                        & 500       & 1,06E-04   & 1,23E+02   & 7,19E+00   & 2,66E+01  \\ \hline
\multirow[t]{10}{*}{ellipsoid}             & 1         & 0.00E+00   & 5,83E-08   & 3,75E-09   & 1,18E-08  \\ \cline{2-6} 
                                        & 2         & 9,27E-13   & 1,51E-04   & 6,41E-06   & 2,69E-05  \\ \cline{2-6} 
                                        & 5         & 1,55E-11   & 1,19E-03   & 8,32E-05   & 2,59E-04  \\ \cline{2-6} 
                                        & 10        & 1,49E-09   & 4,68E-01   & 1,57E-02   & 8,40E-02  \\ \cline{2-6} 
                                        & 20        & 1,11E-08   & 1,03E+00   & 3,52E-02   & 1,85E-01  \\ \cline{2-6} 
                                        & 25        & 4,93E-11   & 1,76E-01   & 8,67E-03   & 3,18E-02  \\ \cline{2-6} 
                                        & 50        & 2,39E-08   & 2,08E+00   & 8,25E-02   & 3,73E-01  \\ \cline{2-6} 
                                        & 100       & 9,82E-07   & 3,55E-01   & 2,84E-02   & 7,47E-02  \\ \cline{2-6} 
                                        & 200       & 6,24E-06   & 4,31E+00   & 2,05E-01   & 7,75E-01  \\ \cline{2-6} 
                                        & 500       & 2,37E-06   & 4,15E+02   & 1,67E+01   & 7,46E+01  \\ \hline
\multirow[t]{10}{*}{elliptic}              & 1         & 0.00E+00   & 3,35E+03   & 2,27E+02   & 7,45E+02  \\ \cline{2-6} 
                                        & 2         & 8,39E-10   & 1,83E+00   & 1,82E-01   & 3,71E-01  \\ \cline{2-6} 
                                        & 5         & 2,17E-05   & 4,07E+01   & 3,30E+00   & 8,35E+00  \\ \cline{2-6} 
                                        & 10        & 1,85E-06   & 5,27E+02   & 1,87E+01   & 9,43E+01  \\ \cline{2-6} 
                                        & 20        & 7,73E-06   & 3,74E+02   & 1,62E+01   & 6,69E+01  \\ \cline{2-6} 
                                        & 25        & 7,21E-06   & 1,51E+03   & 7,83E+01   & 2,84E+02  \\ \cline{2-6} 
                                        & 50        & 5,15E-07   & 2,52E+03   & 1,08E+02   & 4,53E+02  \\ \cline{2-6} 
                                        & 100       & 2,50E-07   & 6,10E+02   & 5,24E+01   & 1,27E+02  \\ \cline{2-6} 
                                        & 200       & 5,00E-04   & 2,86E+03   & 1,67E+02   & 5,39E+02  \\ \cline{2-6} 
                                        & 500       & 3,39E-05   & 1,35E+04   & 6,70E+02   & 2,46E+03  \\ \hline
\multirow[t]{10}{*}{expanded schaffer f6}  & 1         & 0.00E+00   & 1,18E-02   & 2,02E-03   & 3,67E-03  \\ \cline{2-6} 
                                        & 2         & 0.00E+00   & 1,61E-01   & 3,51E-02   & 3,18E-02  \\ \cline{2-6} 
                                        & 5         & 3,01E-02   & 1,94E+00   & 8,00E-01   & 4,77E-01  \\ \cline{2-6} 
                                        & 10        & 2,05E-01   & 4,28E+00   & 2,75E+00   & 8,06E-01  \\ \cline{2-6} 
                                        & 20        & 2,54E-01   & 8,81E+00   & 5,98E+00   & 2,00E+00  \\ \cline{2-6} 
                                        & 25        & 6,85E-02   & 1,15E+01   & 7,53E+00   & 3,27E+00  \\ \cline{2-6} 
                                        & 50        & 1,73E-04   & 2,39E+01   & 1,73E+01   & 6,46E+00  \\ \cline{2-6} 
                                        & 100       & 2,94E-02   & 4,78E+01   & 3,59E+01   & 1,54E+01  \\ \cline{2-6} 
                                        & 200       & 3,56E+00   & 9,50E+01   & 7,60E+01   & 2,87E+01  \\ \cline{2-6} 
                                        & 500       & 7,99E+00   & 2,41E+02   & 1,98E+02   & 6,49E+01  \\ \hline
\multirow[t]{10}{*}{griewank}              & 1         & 0.00E+00   & 3,23E-02   & 4,51E-03   & 7,32E-03  \\ \cline{2-6} 
                                        & 2         & 1,63E-09   & 1,34E-01   & 3,46E-02   & 3,02E-02  \\ \cline{2-6} 
                                        & 5         & 1,16E-09   & 6,40E-01   & 1,93E-01   & 2,13E-01  \\ \cline{2-6} 
                                        & 10        & 6,05E-09   & 1,66E-04   & 1,39E-05   & 3,71E-05  \\ \cline{2-6} 
                                        & 20        & 9,82E-10   & 1,92E-04   & 9,73E-06   & 3,45E-05  \\ \cline{2-6} 
                                        & 25        & 5,18E-12   & 5,52E-05   & 4,49E-06   & 1,20E-05  \\ \cline{2-6} 
                                        & 50        & 3,62E-10   & 8,29E-04   & 3,20E-05   & 1,48E-04  \\ \cline{2-6} 
                                        & 100       & 0.00E+00   & 9,77E-05   & 1,01E-05   & 2,48E-05  \\ \cline{2-6} 
                                        & 200       & 1,61E-09   & 6,94E-05   & 9,49E-06   & 2,00E-05  \\ \cline{2-6} 
                                        & 500       & 4,66E-09   & 2,21E-04   & 1,83E-05   & 4,42E-05  \\ \hline
\multirow[t]{10}{*}{happycat}              & 1         & 1,16E-02   & 2,76E-01   & 1,17E-01   & 6,90E-02  \\ \cline{2-6} 
                                        & 2         & 7,92E-02   & 5,96E-01   & 2,73E-01   & 1,42E-01  \\ \cline{2-6} 
                                        & 5         & 1,04E-01   & 1,23E+00   & 4,92E-01   & 2,83E-01  \\ \cline{2-6} 
                                        & 10        & 1,56E-01   & 1,13E+00   & 6,30E-01   & 2,68E-01  \\ \cline{2-6} 
                                        & 20        & 3,81E-01   & 1,38E+00   & 7,69E-01   & 2,79E-01  \\ \cline{2-6} 
                                        & 25        & 2,37E-01   & 1,42E+00   & 6,72E-01   & 2,97E-01  \\ \cline{2-6} 
                                        & 50        & 2,91E-01   & 1,56E+00   & 8,76E-01   & 3,30E-01  \\ \cline{2-6} 
                                        & 100       & 4,13E-01   & 1,43E+00   & 9,34E-01   & 2,88E-01  \\ \cline{2-6} 
                                        & 200       & 5,07E-01   & 2,01E+00   & 1,16E+00   & 3,69E-01  \\ \cline{2-6} 
                                        & 500       & 3,75E-01   & 2,17E+00   & 1,39E+00   & 4,15E-01  \\ \hline
\multirow[t]{10}{*}{hgbat}                 & 1         & 2,59E-04   & 1,66E-01   & 2,12E-02   & 3,10E-02  \\ \cline{2-6} 
                                        & 2         & 1,18E-02   & 5,01E-01   & 2,58E-01   & 1,66E-01  \\ \cline{2-6} 
                                        & 5         & 6,84E-02   & 5,30E-01   & 4,64E-01   & 9,68E-02  \\ \cline{2-6} 
                                        & 10        & 3,89E-01   & 5,50E-01   & 4,96E-01   & 2,60E-02  \\ \cline{2-6} 
                                        & 20        & 4,99E-01   & 6,33E-01   & 5,18E-01   & 3,25E-02  \\ \cline{2-6} 
                                        & 25        & 3,29E-01   & 9,76E-01   & 5,19E-01   & 9,60E-02  \\ \cline{2-6} 
                                        & 50        & 4,99E-01   & 1,88E+00   & 6,08E-01   & 2,87E-01  \\ \cline{2-6} 
                                        & 100       & 5,00E-01   & 1,31E+00   & 5,70E-01   & 1,69E-01  \\ \cline{2-6} 
                                        & 200       & 5,00E-01   & 1,88E+00   & 6,21E-01   & 2,91E-01  \\ \cline{2-6} 
                                        & 500       & 5,01E-01   & 5,54E+00   & 8,74E-01   & 9,94E-01  \\ \hline
\multirow[t]{10}{*}{katsuura}              & 1         & 0.00E+00   & 5,64E-01   & 6,33E-02   & 1,49E-01  \\ \cline{2-6} 
                                        & 2         & 0.00E+00   & 2,58E+00   & 5,51E-01   & 8,79E-01  \\ \cline{2-6} 
                                        & 5         & 0.00E+00   & 2,63E+00   & 2,89E-01   & 7,50E-01  \\ \cline{2-6} 
                                        & 10        & 0.00E+00   & 1,49E+00   & 9,70E-02   & 3,34E-01  \\ \cline{2-6} 
                                        & 20        & 0.00E+00   & 5,45E+00   & 3,33E-01   & 1,25E+00  \\ \cline{2-6} 
                                        & 25        & 0.00E+00   & 4,72E+00   & 1,60E-01   & 8,47E-01  \\ \cline{2-6} 
                                        & 50        & 0.00E+00   & 5,83E+00   & 1,96E-01   & 1,05E+00  \\ \cline{2-6} 
                                        & 100       & 0.00E+00   & 3,87E-04   & 3,07E-05   & 7,44E-05  \\ \cline{2-6} 
                                        & 200       & 0.00E+00   & 2,20E-05   & 2,74E-06   & 5,80E-06  \\ \cline{2-6} 
                                        & 500       & 0.00E+00   & 1,31E-05   & 9,11E-07   & 2,38E-06  \\ \hline
\multirow[t]{10}{*}{levy}                  & 1         & 3,84E-18   & 1,20E-06   & 9,61E-08   & 2,72E-07  \\ \cline{2-6} 
                                        & 2         & 5,05E-06   & 1,07E-01   & 2,44E-02   & 3,12E-02  \\ \cline{2-6} 
                                        & 5         & 5,40E-04   & 5,77E-01   & 1,20E-01   & 1,10E-01  \\ \cline{2-6} 
                                        & 10        & 2,75E-04   & 3,72E-01   & 1,27E-01   & 1,11E-01  \\ \cline{2-6} 
                                        & 20        & 7,06E-04   & 9,56E-01   & 3,00E-01   & 2,76E-01  \\ \cline{2-6} 
                                        & 25        & 1,11E-02   & 7,52E-01   & 2,87E-01   & 2,42E-01  \\ \cline{2-6} 
                                        & 50        & 3,95E-04   & 1,96E+00   & 6,28E-01   & 5,13E-01  \\ \cline{2-6} 
                                        & 100       & 3,80E-03   & 5,18E+00   & 1,21E+00   & 1,29E+00  \\ \cline{2-6} 
                                        & 200       & 1,49E-03   & 8,73E+00   & 2,59E+00   & 2,31E+00  \\ \cline{2-6} 
                                        & 500       & 7,62E-02   & 1,91E+01   & 6,28E+00   & 5,23E+00  \\ \hline
\multirow[t]{10}{*}{rastrigin}             & 1         & 0.00E+00   & 5,82E-03   & 4,46E-04   & 1,43E-03  \\ \cline{2-6} 
                                        & 2         & 6,57E-09   & 5,01E+00   & 1,15E+00   & 1,18E+00  \\ \cline{2-6} 
                                        & 5         & 5,58E-08   & 2,77E+01   & 5,14E+00   & 8,62E+00  \\ \cline{2-6} 
                                        & 10        & 9,67E-07   & 8,59E+01   & 1,16E+01   & 2,22E+01  \\ \cline{2-6} 
                                        & 20        & 5,17E-07   & 9,68E+01   & 7,61E+00   & 2,13E+01  \\ \cline{2-6} 
                                        & 25        & 5,62E-06   & 4,23E+01   & 4,14E+00   & 1,17E+01  \\ \cline{2-6} 
                                        & 50        & 5,17E-07   & 3,70E+02   & 1,29E+01   & 6,63E+01  \\ \cline{2-6} 
                                        & 100       & 1,64E-06   & 8,90E+01   & 8,40E+00   & 2,34E+01  \\ \cline{2-6} 
                                        & 200       & 1,80E-06   & 5,55E+01   & 2,03E+00   & 9,96E+00  \\ \cline{2-6} 
                                        & 500       & 1,73E-04   & 1,05E+01   & 4,83E-01   & 1,88E+00  \\ \hline
\multirow[t]{10}{*}{rosenbrock}            & 1         & 0.00E+00   & 0.00E+00   & 0.00E+00   & 0.00E+00  \\ \cline{2-6} 
                                        & 2         & 1,21E-05   & 5,34E+01   & 5,58E+00   & 1,31E+01  \\ \cline{2-6} 
                                        & 5         & 9,59E-06   & 3,96E+00   & 6,85E-01   & 1,20E+00  \\ \cline{2-6} 
                                        & 10        & 1,62E-03   & 8,96E+00   & 2,03E+00   & 3,02E+00  \\ \cline{2-6} 
                                        & 20        & 1,74E-04   & 1,88E+01   & 3,83E+00   & 5,20E+00  \\ \cline{2-6} 
                                        & 25        & 9,99E-03   & 2,38E+01   & 4,51E+00   & 7,72E+00  \\ \cline{2-6} 
                                        & 50        & 8,66E-03   & 4,86E+01   & 1,13E+01   & 1,77E+01  \\ \cline{2-6} 
                                        & 100       & 6,96E-05   & 1,00E+02   & 3,76E+01   & 4,14E+01  \\ \cline{2-6} 
                                        & 200       & 3,21E-01   & 2,00E+02   & 6,24E+01   & 7,30E+01  \\ \cline{2-6} 
                                        & 500       & 1,47E-01   & 4,96E+02   & 1,47E+02   & 1,85E+02  \\ \hline
\multirow[t]{10}{*}{schwefel1.2 problem}   & 1         & 0.00E+00   & 2,49E-05   & 8,33E-07   & 4,47E-06  \\ \cline{2-6} 
                                        & 2         & 2,65E-11   & 4,44E-04   & 2,01E-05   & 7,94E-05  \\ \cline{2-6} 
                                        & 5         & 6,70E-09   & 4,48E-03   & 2,25E-04   & 8,03E-04  \\ \cline{2-6} 
                                        & 10        & 4,19E-11   & 1,76E-02   & 1,80E-03   & 4,46E-03  \\ \cline{2-6} 
                                        & 20        & 4,30E-08   & 1,36E-01   & 7,45E-03   & 2,54E-02  \\ \cline{2-6} 
                                        & 25        & 9,39E-08   & 4,02E-02   & 4,59E-03   & 9,97E-03  \\ \cline{2-6} 
                                        & 50        & 1,59E-06   & 1,13E+00   & 7,86E-02   & 2,11E-01  \\ \cline{2-6} 
                                        & 100       & 3,04E-08   & 1,08E+01   & 4,81E-01   & 1,94E+00  \\ \cline{2-6} 
                                        & 200       & 4,42E-08   & 3,48E+02   & 1,44E+01   & 6,22E+01  \\ \cline{2-6} 
                                        & 500       & 9,23E-06   & 5,99E+02   & 4,69E+01   & 1,32E+02  \\ \hline
\multirow[t]{10}{*}{schwefel 2.13 problem} & 1         & 4,40E-09   & 4,99E-01   & 3,79E-02   & 1,11E-01  \\ \cline{2-6} 
                                        & 2         & 9,10E-01   & 3,44E+02   & 3,11E+01   & 6,48E+01  \\ \cline{2-6} 
                                        & 5         & 2,33E+02   & 2,04E+04   & 7,66E+03   & 4,42E+03  \\ \cline{2-6} 
                                        & 10        & 4,57E+04   & 1,54E+05   & 9,48E+04   & 2,60E+04  \\ \cline{2-6} 
                                        & 20        & 5,87E+05   & 1,26E+06   & 8,74E+05   & 1,79E+05  \\ \cline{2-6} 
                                        & 25        & 9,25E+05   & 3,09E+06   & 2,00E+06   & 4,86E+05  \\ \cline{2-6} 
                                        & 50        & 6,35E+06   & 1,03E+07   & 8,20E+06   & 9,65E+05  \\ \cline{2-6} 
                                        & 100       & 3,47E+07   & 4,62E+07   & 4,16E+07   & 3,16E+06  \\ \cline{2-6} 
                                        & 200       & 1,75E+08   & 2,22E+08   & 2,01E+08   & 1,02E+07  \\ \cline{2-6} 
                                        & 500       & 1,32E+09   & 1,48E+09   & 1,41E+09   & 4,38E+07  \\ \hline
\multirow[t]{10}{*}{shubert}               & 1         & -1,45E+01  & -1,43E+01  & -1,45E+01  & 4,55E-02  \\ \cline{2-6} 
                                        & 2         & -2,10E+02  & -1,15E+02  & -1,82E+02  & 2,67E+01  \\ \cline{2-6} 
                                        & 5         & -4,67E+05  & -1,26E+04  & -6,06E+04  & 8,60E+04  \\ \cline{2-6} 
                                        & 10        & -7,43E+10  & -8,99E+06  & -5,02E+09  & 1,49E+10  \\ \cline{2-6} 
                                        & 20        & -8,86E+22  & -1,15E+12  & -4,35E+21  & 1,71E+22  \\ \cline{2-6} 
                                        & 25        & -1,08E+29  & -4,11E+14  & -3,60E+27  & 1,94E+28  \\ \cline{2-6} 
                                        & 50        & -1,11E+58  & -1,50E+27  & -4,75E+56  & 2,06E+57  \\ \cline{2-6} 
                                        & 100       & -1,07E+116 & -2,14E+50  & -4,15E+114 & 1,94E+115 \\ \cline{2-6} 
                                        & 200       & -4,66E+231 & -1,73E+96  & -1,55E+230 & INF       \\ \cline{2-6} 
                                        & 500       & -INF       & -3,23E+231 & -INF       &           \\ \hline
\multirow[t]{10}{*}{sphere}                & 1         & 0.00E+00   & 6,88E-07   & 3,97E-08   & 1,48E-07  \\ \cline{2-6} 
                                        & 2         & 4,80E-09   & 8,08E-05   & 6,38E-06   & 1,58E-05  \\ \cline{2-6} 
                                        & 5         & 1,92E-13   & 2,86E-03   & 1,00E-04   & 5,13E-04  \\ \cline{2-6} 
                                        & 10        & 1,01E-09   & 3,80E-03   & 1,84E-04   & 7,01E-04  \\ \cline{2-6} 
                                        & 20        & 3,20E-10   & 1,15E-03   & 4,19E-05   & 2,07E-04  \\ \cline{2-6} 
                                        & 25        & 1,51E-09   & 8,81E-05   & 1,67E-05   & 2,78E-05  \\ \cline{2-6} 
                                        & 50        & 5,01E-11   & 1,33E-02   & 5,28E-04   & 2,40E-03  \\ \cline{2-6} 
                                        & 100       & 1,91E-09   & 3,11E-03   & 1,53E-04   & 5,62E-04  \\ \cline{2-6} 
                                        & 200       & 2,53E-10   & 1,20E-02   & 8,97E-04   & 2,71E-03  \\ \cline{2-6} 
                                        & 500       & 6,71E-09   & 4,24E-02   & 1,83E-03   & 7,60E-03  \\ \hline
\multirow[t]{10}{*}{step function}         & 1         & 0.00E+00   & 0.00E+00   & 0.00E+00   & 0.00E+00  \\ \cline{2-6} 
                                        & 2         & 0.00E+00   & 0.00E+00   & 0.00E+00   & 0.00E+00  \\ \cline{2-6} 
                                        & 5         & 0.00E+00   & 0.00E+00   & 0.00E+00   & 0.00E+00  \\ \cline{2-6} 
                                        & 10        & 0.00E+00   & 0.00E+00   & 0.00E+00   & 0.00E+00  \\ \cline{2-6} 
                                        & 20        & 0.00E+00   & 0.00E+00   & 0.00E+00   & 0.00E+00  \\ \cline{2-6} 
                                        & 25        & 0.00E+00   & 0.00E+00   & 0.00E+00   & 0.00E+00  \\ \cline{2-6} 
                                        & 50        & 0.00E+00   & 0.00E+00   & 0.00E+00   & 0.00E+00  \\ \cline{2-6} 
                                        & 100       & 0.00E+00   & 0.00E+00   & 0.00E+00   & 0.00E+00  \\ \cline{2-6} 
                                        & 200       & 0.00E+00   & 0.00E+00   & 0.00E+00   & 0.00E+00  \\ \cline{2-6} 
                                        & 500       & 0.00E+00   & 0.00E+00   & 0.00E+00   & 0.00E+00  \\ \hline
\multirow[t]{10}{*}{vincent}               & 1         & -1,00E+00  & -1,00E+00  & -1,00E+00  & 9,70E-05  \\ \cline{2-6} 
                                        & 2         & -1,00E+00  & -9,14E-01  & -9,81E-01  & 2,51E-02  \\ \cline{2-6} 
                                        & 5         & -1,00E+00  & -6,82E-01  & -8,87E-01  & 8,80E-02  \\ \cline{2-6} 
                                        & 10        & -9,98E-01  & -3,11E-01  & -7,95E-01  & 1,55E-01  \\ \cline{2-6} 
                                        & 20        & -9,81E-01  & -3,15E-01  & -7,91E-01  & 1,53E-01  \\ \cline{2-6} 
                                        & 25        & -9,97E-01  & -3,88E-01  & -7,99E-01  & 1,58E-01  \\ \cline{2-6} 
                                        & 50        & -9,99E-01  & -2,16E-01  & -7,20E-01  & 1,89E-01  \\ \cline{2-6} 
                                        & 100       & -9,99E-01  & -3,11E-01  & -7,71E-01  & 2,05E-01  \\ \cline{2-6} 
                                        & 200       & -9,99E-01  & -3,94E-01  & -7,68E-01  & 1,74E-01  \\ \cline{2-6} 
                                        & 500       & -1,00E+00  & -4,08E-01  & -7,89E-01  & 1,44E-01  \\ \hline
\multirow[t]{10}{*}{weierstrass}           & 1         & 0.00E+00   & 1,42E-01   & 1,76E-02   & 3,22E-02  \\ \cline{2-6} 
                                        & 2         & 0.00E+00   & 9,98E-01   & 2,73E-01   & 3,56E-01  \\ \cline{2-6} 
                                        & 5         & 0.00E+00   & 4,82E+00   & 5,23E-01   & 1,35E+00  \\ \cline{2-6} 
                                        & 10        & 0.00E+00   & 1,23E+01   & 1,26E+00   & 3,31E+00  \\ \cline{2-6} 
                                        & 20        & 0.00E+00   & 2,81E+01   & 1,99E+00   & 6,26E+00  \\ \cline{2-6} 
                                        & 25        & 3,55E-14   & 3,57E+01   & 1,36E+00   & 6,41E+00  \\ \cline{2-6} 
                                        & 50        & 4,26E-14   & 7,86E+01   & 3,01E+00   & 1,41E+01  \\ \cline{2-6} 
                                        & 100       & 3,41E-13   & 1,75E+02   & 8,60E+00   & 3,26E+01  \\ \cline{2-6} 
                                        & 200       & 5,68E-14   & 2,49E+02   & 8,36E+00   & 4,48E+01  \\ \cline{2-6} 
                                        & 500       & 2,50E-12   & 1,99E+00   & 3,14E-01   & 6,10E-01  \\ \hline
\multirow[t]{10}{*}{zakharov}              & 1         & 0.00E+00   & 2,71E-07   & 1,09E-08   & 4,91E-08  \\ \cline{2-6} 
                                        & 2         & 0.00E+00   & 1,99E-03   & 9,87E-05   & 3,83E-04  \\ \cline{2-6} 
                                        & 5         & 2,34E-12   & 6,24E-03   & 2,46E-04   & 1,12E-03  \\ \cline{2-6} 
                                        & 10        & 2,12E-09   & 1,31E-02   & 6,41E-04   & 2,37E-03  \\ \cline{2-6} 
                                        & 20        & 4,35E-09   & 3,66E-02   & 1,39E-03   & 6,56E-03  \\ \cline{2-6} 
                                        & 25        & 1,78E-09   & 5,35E-02   & 4,72E-03   & 1,18E-02  \\ \cline{2-6} 
                                        & 50        & 8,89E-08   & 1,18E-02   & 8,86E-04   & 2,32E-03  \\ \cline{2-6} 
                                        & 100       & 8,83E-08   & 1,23E-01   & 9,77E-03   & 2,52E-02  \\ \cline{2-6} 
                                        & 200       & 1,99E-06   & 1,77E+00   & 1,34E-01   & 3,43E-01  \\ \cline{2-6} 
                                        & 500       & 1,14E-09   & 7,12E+02   & 2,95E+01   & 1,29E+02  \\ \hline
\end{longtable}

\begin{figure}[H]
\centering
\begin{tikzpicture}
  \begin{axis}[
    title={Hasil \textit{benchmarking} fungsi Ackley},
    legend entries={{KMA Best}, {KMA Worst}, {KMA Avg}, {EHO Best}, {EHO Worst}, {EHO Avg}},
    legend pos = outer north east,
    xlabel = Dimensi,
    ylabel = Nilai solusi,
    symbolic x coords={1,2,5,10,20,25,50,100,200,500},
    xtick=data,
    x tick label style={rotate=45, anchor=east},
    enlargelimits=0.05
    ]
    
    % KMA Best
    \addplot[blue,mark=diamond*] table[
      col sep=tab,
      skip first n=1,
      x index=1,
      y index=2,
    ] {data/kma_ackley.tsv};
    
    % KMA Worst
    \addplot[green,mark=diamond*] table[
      col sep=tab,
      skip first n=1,
      x index=1,
      y index=3,
    ] {data/kma_ackley.tsv};
    
    % KMA Avg
    \addplot[orange,mark=diamond*] table[
      col sep=tab,
      skip first n=1,
      x index=1,
      y index=4,
    ] {data/kma_ackley.tsv};
    
    % EHO Best
    \addplot[red,mark=square*] table[
      col sep=tab,
      skip first n=1,
      x index=1,
      y index=2,
    ] {data/eho_ackley.tsv};
    
    % EHO Worst
    \addplot[magenta,mark=square*] table[
      col sep=tab,
      skip first n=1,
      x index=1,
      y index=3,
    ] {data/eho_ackley.tsv};
    
    % EHO Avg
    \addplot[cyan,mark=square*] table[
      col sep=tab,
      skip first n=1,
      x index=1,
      y index=4,
    ] {data/eho_ackley.tsv};
    
  \end{axis}
\end{tikzpicture}
\caption{Grafik hasil \textit{benchmarking} fungsi Ackley}
\label{fig:graph_ackley}
\end{figure}

EHO menunjukkan performa yang sangat stabil dan konsisten di berbagai dimensi, selalu mendekati nilai optimum, sehingga lebih dapat diandalkan untuk penyelesaian fungsi Ackley. Sementara itu, KMA mampu menemukan solusi optimal pada percobaan tertentu (\textit{best}), tetapi memiliki variasi hasil yang tinggi (\textit{avg} dan \textit{worst}), terutama pada dimensi 20, 50, dan 100. \cref{fig:graph_ackley} mengonfirmasi bahwa stabilitas EHO lebih baik dibandingkan KMA dalam menyelesaikan fungsi Ackley, meskipun KMA tetap memiliki potensi mencapai solusi optimal, tetapi dengan stabilitas yang lebih rendah.

\begin{figure}[H]
\centering
\begin{tikzpicture}
  \begin{axis}[
    title={Hasil \textit{benchmarking} fungsi Bent Cigar},
    legend entries={{KMA Best}, {KMA Worst}, {KMA Avg}, {EHO Best}, {EHO Worst}, {EHO Avg}},
    legend pos = outer north east,
    xlabel = Dimensi,
    ylabel = Nilai solusi,
    symbolic x coords={1,2,5,10,20,25,50,100,200,500},
    xtick=data,
    x tick label style={rotate=45, anchor=east},
    enlargelimits=0.05
    ]
    
    % KMA Best
    \addplot[blue,mark=diamond*] table[
      col sep=tab,
      skip first n=1,
      x index=1,
      y index=2,
    ] {data/kma_bent_cigar.tsv};
    
    % KMA Worst
    \addplot[green,mark=diamond*] table[
      col sep=tab,
      skip first n=1,
      x index=1,
      y index=3,
    ] {data/kma_bent_cigar.tsv};
    
    % KMA Avg
    \addplot[orange,mark=diamond*] table[
      col sep=tab,
      skip first n=1,
      x index=1,
      y index=4,
    ] {data/kma_bent_cigar.tsv};
    
    % EHO Best
    \addplot[red,mark=square*] table[
      col sep=tab,
      skip first n=1,
      x index=1,
      y index=2,
    ] {data/eho_bent_cigar.tsv};
    
    % EHO Worst
    \addplot[magenta,mark=square*] table[
      col sep=tab,
      skip first n=1,
      x index=1,
      y index=3,
    ] {data/eho_bent_cigar.tsv};
    
    % EHO Avg
    \addplot[cyan,mark=square*] table[
      col sep=tab,
      skip first n=1,
      x index=1,
      y index=4,
    ] {data/eho_bent_cigar.tsv};
    
  \end{axis}
\end{tikzpicture}
\caption{Grafik hasil \textit{benchmarking} fungsi Bent Cigar}
\label{fig:graph_bent_cigar}
\end{figure}

Baik EHO maupun KMA menunjukkan performa yang baik pada dimensi rendah dengan nilai solusi mendekati optimum. Namun, pada dimensi tinggi, stabilitas kedua algoritma mengalami penurunan, terlihat dari kenaikan signifikan pada nilai \textit{worst} dan \textit{avg}, terutama pada dimensi 500. Meski demikian, nilai \textit{best} yang tetap mendekati nol pada kedua algoritma menunjukkan bahwa potensi untuk mencapai solusi optimal masih ada, walaupun diiringi dengan peningkatan ketidakstabilan pada fungsi Bent Cigar.

\begin{figure}[H]
\centering
\begin{tikzpicture}
  \begin{axis}[
    title={Hasil \textit{benchmarking} fungsi Different Power},
    legend entries={{KMA Best}, {KMA Worst}, {KMA Avg}, {EHO Best}, {EHO Worst}, {EHO Avg}},
    legend pos = outer north east,
    xlabel = Dimensi,
    ylabel = Nilai solusi,
    symbolic x coords={1,2,5,10,20,25,50,100,200,500},
    xtick=data,
    x tick label style={rotate=45, anchor=east},
    enlargelimits=0.05
    ]
    
    % KMA Best
    \addplot[blue,mark=diamond*] table[
      col sep=tab,
      skip first n=1,
      x index=1,
      y index=2,
    ] {data/kma_different_power.tsv};
    
    % KMA Worst
    \addplot[green,mark=diamond*] table[
      col sep=tab,
      skip first n=1,
      x index=1,
      y index=3,
    ] {data/kma_different_power.tsv};
    
    % KMA Avg
    \addplot[orange,mark=diamond*] table[
      col sep=tab,
      skip first n=1,
      x index=1,
      y index=4,
    ] {data/kma_different_power.tsv};
    
    % EHO Best
    \addplot[red,mark=square*] table[
      col sep=tab,
      skip first n=1,
      x index=1,
      y index=2,
    ] {data/eho_different_power.tsv};
    
    % EHO Worst
    \addplot[magenta,mark=square*] table[
      col sep=tab,
      skip first n=1,
      x index=1,
      y index=3,
    ] {data/eho_different_power.tsv};
    
    % EHO Avg
    \addplot[cyan,mark=square*] table[
      col sep=tab,
      skip first n=1,
      x index=1,
      y index=4,
    ] {data/eho_different_power.tsv};
    
  \end{axis}
\end{tikzpicture}
\caption{Grafik hasil \textit{benchmarking} fungsi Different Power}
\label{fig:graph_different_power}
\end{figure}

Baik EHO maupun KMA dapat menangani fungsi Different Power dengan baik pada dimensi rendah hingga sedang, dengan nilai solusi mendekati optimal. Namun, pada dimensi tinggi, EHO menunjukkan kenaikan yang lebih signifikan pada nilai \textit{worst} dan \textit{avg} dibandingkan KMA, menunjukkan penurunan stabilitas pada dimensi tinggi. KMA cenderung lebih stabil pada dimensi tinggi daripada EHO untuk fungsi ini. Meskipun demikian, nilai \textit{best} kedua algoritma tetap optimal, mengindikasikan bahwa potensi mencapai solusi terbaik tetap ada pada keduanya meskipun variasi nilai solusi meningkat.

\begin{figure}[H]
\centering
\begin{tikzpicture}
  \begin{axis}[
    title={Hasil \textit{benchmarking} fungsi Discus},
    legend entries={{KMA Best}, {KMA Worst}, {KMA Avg}, {EHO Best}, {EHO Worst}, {EHO Avg}},
    legend pos = outer north east,
    xlabel = Dimensi,
    ylabel = Nilai solusi,
    symbolic x coords={1,2,5,10,20,25,50,100,200,500},
    xtick=data,
    x tick label style={rotate=45, anchor=east},
    enlargelimits=0.05
    ]
    
    % KMA Best
    \addplot[blue,mark=diamond*] table[
      col sep=tab,
      skip first n=1,
      x index=1,
      y index=2,
    ] {data/kma_discus.tsv};
    
    % KMA Worst
    \addplot[green,mark=diamond*] table[
      col sep=tab,
      skip first n=1,
      x index=1,
      y index=3,
    ] {data/kma_discus.tsv};
    
    % KMA Avg
    \addplot[orange,mark=diamond*] table[
      col sep=tab,
      skip first n=1,
      x index=1,
      y index=4,
    ] {data/kma_discus.tsv};
    
    % EHO Best
    \addplot[red,mark=square*] table[
      col sep=tab,
      skip first n=1,
      x index=1,
      y index=2,
    ] {data/eho_discus.tsv};
    
    % EHO Worst
    \addplot[magenta,mark=square*] table[
      col sep=tab,
      skip first n=1,
      x index=1,
      y index=3,
    ] {data/eho_discus.tsv};
    
    % EHO Avg
    \addplot[cyan,mark=square*] table[
      col sep=tab,
      skip first n=1,
      x index=1,
      y index=4,
    ] {data/eho_discus.tsv};
    
  \end{axis}
\end{tikzpicture}
\caption{Grafik hasil \textit{benchmarking} fungsi Discus}
\label{fig:graph_discus}
\end{figure}

Pada fungsi Discus, EHO menunjukkan performa yang secara konsisten lebih stabil di semua dimensi dibandingkan KMA. KMA sendiri menunjukkan ketidakstabilan yang signifikan pada dimensi tinggi (500), dengan nilai \textit{worst} yang sangat besar, mengindikasikan kesulitan algoritma ini dalam mempertahankan kualitas solusi pada kondisi tertentu. Meskipun demikian, nilai \textit{best} kedua algoritma tetap sangat baik di semua dimensi, menunjukkan bahwa keduanya memiliki potensi untuk menemukan solusi optimal meskipun ketidakstabilan muncul pada kondisi \textit{worst} di dimensi tinggi. Secara keseluruhan, EHO terbukti lebih konsisten daripada KMA dalam menangani fungsi Discus di seluruh variasi dimensi.

\begin{figure}[H]
\centering
\begin{tikzpicture}
  \begin{axis}[
    title={Hasil \textit{benchmarking} fungsi Ellipsoid},
    legend entries={{KMA Best}, {KMA Worst}, {KMA Avg}, {EHO Best}, {EHO Worst}, {EHO Avg}},
    legend pos = outer north east,
    xlabel = Dimensi,
    ylabel = Nilai solusi,
    symbolic x coords={1,2,5,10,20,25,50,100,200,500},
    xtick=data,
    x tick label style={rotate=45, anchor=east},
    enlargelimits=0.05
    ]
    
    % KMA Best
    \addplot[blue,mark=diamond*] table[
      col sep=tab,
      skip first n=1,
      x index=1,
      y index=2,
    ] {data/kma_ellipsoid.tsv};
    
    % KMA Worst
    \addplot[green,mark=diamond*] table[
      col sep=tab,
      skip first n=1,
      x index=1,
      y index=3,
    ] {data/kma_ellipsoid.tsv};
    
    % KMA Avg
    \addplot[orange,mark=diamond*] table[
      col sep=tab,
      skip first n=1,
      x index=1,
      y index=4,
    ] {data/kma_ellipsoid.tsv};
    
    % EHO Best
    \addplot[red,mark=square*] table[
      col sep=tab,
      skip first n=1,
      x index=1,
      y index=2,
    ] {data/eho_ellipsoid.tsv};
    
    % EHO Worst
    \addplot[magenta,mark=square*] table[
      col sep=tab,
      skip first n=1,
      x index=1,
      y index=3,
    ] {data/eho_ellipsoid.tsv};
    
    % EHO Avg
    \addplot[cyan,mark=square*] table[
      col sep=tab,
      skip first n=1,
      x index=1,
      y index=4,
    ] {data/eho_ellipsoid.tsv};
    
  \end{axis}
\end{tikzpicture}
\caption{Grafik hasil \textit{benchmarking} fungsi Ellipsoid}
\label{fig:graph_ellipsoid}
\end{figure}

Pada fungsi Ellipsoid, EHO menunjukkan kestabilan yang lebih baik dibanding KMA, terutama pada dimensi tinggi. KMA mengalami degradasi performa yang signifikan pada kondisi terburuk saat dimensi mencapai 500, ditandai dengan lonjakan drastis pada nilai \textit{worst}. Meskipun demikian, nilai \textit{best} kedua algoritma tetap rendah, menunjukkan bahwa keduanya masih mampu menemukan solusi optimal. Secara keseluruhan, EHO lebih konsisten dan stabil pada fungsi Ellipsoid, terutama pada dimensi tinggi, sementara KMA lebih cocok digunakan pada dimensi rendah hingga sedang dan membutuhkan perbaikan untuk meningkatkan kestabilannya pada dimensi tinggi.

\begin{figure}[H]
\centering
\begin{tikzpicture}
  \begin{axis}[
    title={Hasil \textit{benchmarking} fungsi Elliptic},
    legend entries={{KMA Best}, {KMA Worst}, {KMA Avg}, {EHO Best}, {EHO Worst}, {EHO Avg}},
    legend pos = outer north east,
    xlabel = Dimensi,
    ylabel = Nilai solusi,
    symbolic x coords={1,2,5,10,20,25,50,100,200,500},
    xtick=data,
    x tick label style={rotate=45, anchor=east},
    enlargelimits=0.05
    ]
    
    % KMA Best
    \addplot[blue,mark=diamond*] table[
      col sep=tab,
      skip first n=1,
      x index=1,
      y index=2,
    ] {data/kma_elliptic.tsv};
    
    % KMA Worst
    \addplot[green,mark=diamond*] table[
      col sep=tab,
      skip first n=1,
      x index=1,
      y index=3,
    ] {data/kma_elliptic.tsv};
    
    % KMA Avg
    \addplot[orange,mark=diamond*] table[
      col sep=tab,
      skip first n=1,
      x index=1,
      y index=4,
    ] {data/kma_elliptic.tsv};
    
    % EHO Best
    \addplot[red,mark=square*] table[
      col sep=tab,
      skip first n=1,
      x index=1,
      y index=2,
    ] {data/eho_elliptic.tsv};
    
    % EHO Worst
    \addplot[magenta,mark=square*] table[
      col sep=tab,
      skip first n=1,
      x index=1,
      y index=3,
    ] {data/eho_elliptic.tsv};
    
    % EHO Avg
    \addplot[cyan,mark=square*] table[
      col sep=tab,
      skip first n=1,
      x index=1,
      y index=4,
    ] {data/eho_elliptic.tsv};
    
  \end{axis}
\end{tikzpicture}
\caption{Grafik hasil \textit{benchmarking} fungsi Elliptic}
\label{fig:graph_elliptic}
\end{figure}

Pada fungsi Elliptic, KMA memiliki kestabilan yang lebih baik dibandingkan EHO pada dimensi tinggi. EHO menunjukkan performa baik pada dimensi rendah, tetapi mengalami degradasi stabilitas signifikan pada kondisi \textit{worst} ketika dimensi mencapai 500. KMA menunjukkan pertumbuhan nilai solusi yang lebih terkendali dibandingkan EHO pada dimensi tinggi, walaupun keduanya mengalami kenaikan. Kedua algoritma mampu menemukan nilai \textit{best} yang rendah meskipun terjadi ketidakstabilan pada nilai \textit{avg} dan kondisi \textit{worst}.

\begin{figure}[H]
\centering
\begin{tikzpicture}
  \begin{axis}[
    title={Hasil \textit{benchmarking} fungsi Expanded Schaffer F6},
    legend entries={{KMA Best}, {KMA Worst}, {KMA Avg}, {EHO Best}, {EHO Worst}, {EHO Avg}},
    legend pos = outer north east,
    xlabel = Dimensi,
    ylabel = Nilai solusi,
    symbolic x coords={1,2,5,10,20,25,50,100,200,500},
    xtick=data,
    x tick label style={rotate=45, anchor=east},
    enlargelimits=0.05
    ]
    
    % KMA Best
    \addplot[blue,mark=diamond*] table[
      col sep=tab,
      skip first n=1,
      x index=1,
      y index=2,
    ] {data/kma_expanded_schaffer_f6.tsv};
    
    % KMA Worst
    \addplot[green,mark=diamond*] table[
      col sep=tab,
      skip first n=1,
      x index=1,
      y index=3,
    ] {data/kma_expanded_schaffer_f6.tsv};
    
    % KMA Avg
    \addplot[orange,mark=diamond*] table[
      col sep=tab,
      skip first n=1,
      x index=1,
      y index=4,
    ] {data/kma_expanded_schaffer_f6.tsv};
    
    % EHO Best
    \addplot[red,mark=square*] table[
      col sep=tab,
      skip first n=1,
      x index=1,
      y index=2,
    ] {data/eho_expanded_schaffer_f6.tsv};
    
    % EHO Worst
    \addplot[magenta,mark=square*] table[
      col sep=tab,
      skip first n=1,
      x index=1,
      y index=3,
    ] {data/eho_expanded_schaffer_f6.tsv};
    
    % EHO Avg
    \addplot[cyan,mark=square*] table[
      col sep=tab,
      skip first n=1,
      x index=1,
      y index=4,
    ] {data/eho_expanded_schaffer_f6.tsv};
    
  \end{axis}
\end{tikzpicture}
\caption{Grafik hasil \textit{benchmarking} fungsi Expanded Schaffer F6}
\label{fig:graph_expanded_schaffer_f6}
\end{figure}

Semakin tinggi dimensi, semakin besar nilai solusi fungsi Expanded Schaffer F6, menunjukkan fungsi ini cukup sensitif terhadap kenaikan dimensi. KMA menunjukkan performa lebih stabil dan konsisten dibandingkan EHO, terutama terlihat pada nilai \textit{best} dan \textit{avg} pada dimensi tinggi. EHO menunjukkan ketidakstabilan pada kondisi \textit{worst} saat dimensi sangat tinggi, tetapi masih mampu menghasilkan nilai \textit{best} yang rendah pada beberapa iterasi. Nilai \textit{avg} EHO tetap lebih tinggi dibandingkan KMA pada dimensi tinggi, menunjukkan rata-rata hasil EHO lebih buruk dalam menyelesaikan fungsi ini pada dimensi tinggi.

\begin{figure}[H]
\centering
\begin{tikzpicture}
  \begin{axis}[
    title={Hasil \textit{benchmarking} fungsi Griewank},
    legend entries={{KMA Best}, {KMA Worst}, {KMA Avg}, {EHO Best}, {EHO Worst}, {EHO Avg}},
    legend pos = outer north east,
    xlabel = Dimensi,
    ylabel = Nilai solusi,
    symbolic x coords={1,2,5,10,20,25,50,100,200,500},
    xtick=data,
    x tick label style={rotate=45, anchor=east},
    enlargelimits=0.05
    ]
    
    % KMA Best
    \addplot[blue,mark=diamond*] table[
      col sep=tab,
      skip first n=1,
      x index=1,
      y index=2,
    ] {data/kma_griewank.tsv};
    
    % KMA Worst
    \addplot[green,mark=diamond*] table[
      col sep=tab,
      skip first n=1,
      x index=1,
      y index=3,
    ] {data/kma_griewank.tsv};
    
    % KMA Avg
    \addplot[orange,mark=diamond*] table[
      col sep=tab,
      skip first n=1,
      x index=1,
      y index=4,
    ] {data/kma_griewank.tsv};
    
    % EHO Best
    \addplot[red,mark=square*] table[
      col sep=tab,
      skip first n=1,
      x index=1,
      y index=2,
    ] {data/eho_griewank.tsv};
    
    % EHO Worst
    \addplot[magenta,mark=square*] table[
      col sep=tab,
      skip first n=1,
      x index=1,
      y index=3,
    ] {data/eho_griewank.tsv};
    
    % EHO Avg
    \addplot[cyan,mark=square*] table[
      col sep=tab,
      skip first n=1,
      x index=1,
      y index=4,
    ] {data/eho_griewank.tsv};
    
  \end{axis}
\end{tikzpicture}
\caption{Grafik hasil \textit{benchmarking} fungsi Griewank}
\label{fig:graph_griewank}
\end{figure}

Fungsi Griewank menunjukkan karakteristik yang relatif stabil dan mudah diselesaikan oleh kedua algoritma, KMA maupun EHO, bahkan ketika dimensi meningkat. Perbedaan performa hanya terlihat pada dimensi sangat rendah (2–10), kemungkinan dipengaruhi oleh sensitivitas awal fungsi dan kondisi populasi awal algoritma. Pada dimensi tinggi, fungsi Griewank tidak memperlihatkan peningkatan kesulitan yang signifikan, sehingga menjadi pilihan yang baik untuk menguji stabilitas algoritma dalam skenario berdimensi tinggi. Baik KMA maupun EHO mampu menemukan solusi optimal dengan nilai mendekati nol secara stabil meskipun dimensi bertambah, menunjukkan bahwa kedua algoritma bekerja sangat baik pada fungsi ini.

\begin{figure}[H]
\centering
\begin{tikzpicture}
  \begin{axis}[
    title={Hasil \textit{benchmarking} fungsi Happycat},
    legend entries={{KMA Best}, {KMA Worst}, {KMA Avg}, {EHO Best}, {EHO Worst}, {EHO Avg}},
    legend pos = outer north east,
    xlabel = Dimensi,
    ylabel = Nilai solusi,
    symbolic x coords={1,2,5,10,20,25,50,100,200,500},
    xtick=data,
    x tick label style={rotate=45, anchor=east},
    enlargelimits=0.05
    ]
    
    % KMA Best
    \addplot[blue,mark=diamond*] table[
      col sep=tab,
      skip first n=1,
      x index=1,
      y index=2,
    ] {data/kma_happycat.tsv};
    
    % KMA Worst
    \addplot[green,mark=diamond*] table[
      col sep=tab,
      skip first n=1,
      x index=1,
      y index=3,
    ] {data/kma_happycat.tsv};
    
    % KMA Avg
    \addplot[orange,mark=diamond*] table[
      col sep=tab,
      skip first n=1,
      x index=1,
      y index=4,
    ] {data/kma_happycat.tsv};
    
    % EHO Best
    \addplot[red,mark=square*] table[
      col sep=tab,
      skip first n=1,
      x index=1,
      y index=2,
    ] {data/eho_happycat.tsv};
    
    % EHO Worst
    \addplot[magenta,mark=square*] table[
      col sep=tab,
      skip first n=1,
      x index=1,
      y index=3,
    ] {data/eho_happycat.tsv};
    
    % EHO Avg
    \addplot[cyan,mark=square*] table[
      col sep=tab,
      skip first n=1,
      x index=1,
      y index=4,
    ] {data/eho_happycat.tsv};
    
  \end{axis}
\end{tikzpicture}
\caption{Grafik hasil \textit{benchmarking} fungsi Happycat}
\label{fig:graph_happycat}
\end{figure}

KMA menunjukkan keunggulan dalam pencapaian solusi optimal dengan nilai terbaik (\textit{Best}) yang lebih rendah dibandingkan EHO pada semua dimensi, membuktikan kemampuannya menemukan solusi lebih optimal untuk fungsi Happycat. Namun, nilai rata-rata (\textit{Avg}) dan terburuk (\textit{Worst}) KMA lebih tinggi daripada EHO, terutama pada dimensi yang lebih besar, mengindikasikan variasi hasil yang lebih lebar dan stabilitas yang lebih rendah. Di sisi lain, EHO mencatat performa yang stabil dengan rata-rata (\textit{Avg}) dan nilai terburuk (\textit{Worst}) yang konsisten, meskipun solusi terbaiknya (\textit{Best}) tetap kurang optimal dibandingkan KMA.

\begin{figure}[H]
\centering
\begin{tikzpicture}
  \begin{axis}[
    title={Hasil \textit{benchmarking} fungsi Hgbat},
    legend entries={{KMA Best}, {KMA Worst}, {KMA Avg}, {EHO Best}, {EHO Worst}, {EHO Avg}},
    legend pos = outer north east,
    xlabel = Dimensi,
    ylabel = Nilai solusi,
    symbolic x coords={1,2,5,10,20,25,50,100,200,500},
    xtick=data,
    x tick label style={rotate=45, anchor=east},
    enlargelimits=0.05
    ]
    
    % KMA Best
    \addplot[blue,mark=diamond*] table[
      col sep=tab,
      skip first n=1,
      x index=1,
      y index=2,
    ] {data/kma_hgbat.tsv};
    
    % KMA Worst
    \addplot[green,mark=diamond*] table[
      col sep=tab,
      skip first n=1,
      x index=1,
      y index=3,
    ] {data/kma_hgbat.tsv};
    
    % KMA Avg
    \addplot[orange,mark=diamond*] table[
      col sep=tab,
      skip first n=1,
      x index=1,
      y index=4,
    ] {data/kma_hgbat.tsv};
    
    % EHO Best
    \addplot[red,mark=square*] table[
      col sep=tab,
      skip first n=1,
      x index=1,
      y index=2,
    ] {data/eho_hgbat.tsv};
    
    % EHO Worst
    \addplot[magenta,mark=square*] table[
      col sep=tab,
      skip first n=1,
      x index=1,
      y index=3,
    ] {data/eho_hgbat.tsv};
    
    % EHO Avg
    \addplot[cyan,mark=square*] table[
      col sep=tab,
      skip first n=1,
      x index=1,
      y index=4,
    ] {data/eho_hgbat.tsv};
    
  \end{axis}
\end{tikzpicture}
\caption{Grafik hasil \textit{benchmarking} fungsi Hgbat}
\label{fig:graph_hgbat}
\end{figure}

Dalam evaluasi fungsi Hgbat, KMA menunjukkan keunggulan dalam menemukan nilai solusi terbaik (\textit{Best}) yang konsisten lebih rendah dibandingkan EHO di semua dimensi, sejalan dengan performa optimalnya. Namun, KMA menunjukkan ketidakstabilan signifikan pada dimensi tinggi (terutama 500), di mana nilai terburuknya (\textit{Worst}) meningkat drastis. Sebaliknya, EHO mempertahankan pola yang lebih stabil, dengan nilai rata-rata dan terburuk yang relatif berdekatan, meskipun nilai terbaik EHO tetap lebih tinggi daripada KMA.

\begin{figure}[H]
\centering
\begin{tikzpicture}
  \begin{axis}[
    title={Hasil \textit{benchmarking} fungsi Katsuura},
    legend entries={{KMA Best}, {KMA Worst}, {KMA Avg}, {EHO Best}, {EHO Worst}, {EHO Avg}},
    legend pos = outer north east,
    xlabel = Dimensi,
    ylabel = Nilai solusi,  
    symbolic x coords={1,2,5,10,20,25,50,100,200,500},
    xtick=data,
    x tick label style={rotate=45, anchor=east},
    enlargelimits=0.05
    ]
    
    % KMA Best
    \addplot[blue,mark=diamond*] table[
      col sep=tab,
      skip first n=1,
      x index=1,
      y index=2,
    ] {data/kma_katsuura.tsv};
    
    % KMA Worst
    \addplot[green,mark=diamond*] table[
      col sep=tab,
      skip first n=1,
      x index=1,
      y index=3,
    ] {data/kma_katsuura.tsv};
    
    % KMA Avg
    \addplot[orange,mark=diamond*] table[
      col sep=tab,
      skip first n=1,
      x index=1,
      y index=4,
    ] {data/kma_katsuura.tsv};
    
    % EHO Best
    \addplot[red,mark=square*] table[
      col sep=tab,
      skip first n=1,
      x index=1,
      y index=2,
    ] {data/eho_katsuura.tsv};
    
    % EHO Worst
    \addplot[magenta,mark=square*] table[
      col sep=tab,
      skip first n=1,
      x index=1,
      y index=3,
    ] {data/eho_katsuura.tsv};
    
    % EHO Avg
    \addplot[cyan,mark=square*] table[
      col sep=tab,
      skip first n=1,
      x index=1,
      y index=4,
    ] {data/eho_katsuura.tsv};
    
  \end{axis}
\end{tikzpicture}
\caption{Grafik hasil \textit{benchmarking} fungsi Katsuura}
\label{fig:graph_katsuura}
\end{figure}

Fungsi Katsuura merupakan fungsi \textit{benchmark} yang sangat menantang, terutama pada dimensi tinggi dan menengah, di mana EHO menunjukkan peningkatan drastis pada nilai solusi. Dalam menghadapi tantangan ini, KMA mampu mempertahankan performa yang lebih baik dan stabil di semua dimensi, sehingga unggul secara signifikan dibandingkan EHO. EHO mengalami fluktuasi dan ketidakkonsistenan, terutama pada metrik \textit{Worst} dan \textit{Avg}, dengan penurunan stabilitas yang terlihat jelas pada rentang dimensi 25–100. Berdasarkan hasil ini, KMA terbukti lebih layak digunakan untuk optimasi fungsi kompleks seperti Katsuura, terutama dalam kondisi dengan jumlah dimensi yang tinggi.

\begin{figure}[H]
\centering
\begin{tikzpicture}
  \begin{axis}[
    title={Hasil \textit{benchmarking} fungsi Levy},
    legend entries={{KMA Best}, {KMA Worst}, {KMA Avg}, {EHO Best}, {EHO Worst}, {EHO Avg}},
    legend pos = outer north east,
    xlabel = Dimensi,
    ylabel = Nilai solusi,
    symbolic x coords={1,2,5,10,20,25,50,100,200,500},
    xtick=data,
    x tick label style={rotate=45, anchor=east},
    enlargelimits=0.05
    ]
    
    % KMA Best
    \addplot[blue,mark=diamond*] table[
      col sep=tab,
      skip first n=1,
      x index=1,
      y index=2,
    ] {data/kma_levy.tsv};
    
    % KMA Worst
    \addplot[green,mark=diamond*] table[
      col sep=tab,
      skip first n=1,
      x index=1,
      y index=3,
    ] {data/kma_levy.tsv};
    
    % KMA Avg
    \addplot[orange,mark=diamond*] table[
      col sep=tab,
      skip first n=1,
      x index=1,
      y index=4,
    ] {data/kma_levy.tsv};
    
    % EHO Best
    \addplot[red,mark=square*] table[
      col sep=tab,
      skip first n=1,
      x index=1,
      y index=2,
    ] {data/eho_levy.tsv};
    
    % EHO Worst
    \addplot[magenta,mark=square*] table[
      col sep=tab,
      skip first n=1,
      x index=1,
      y index=3,
    ] {data/eho_levy.tsv};
    
    % EHO Avg
    \addplot[cyan,mark=square*] table[
      col sep=tab,
      skip first n=1,
      x index=1,
      y index=4,
    ] {data/eho_levy.tsv};
    
  \end{axis}
\end{tikzpicture}
\caption{Grafik hasil \textit{benchmarking} fungsi Levy}
\label{fig:graph_levy}
\end{figure}

Fungsi Levy menjadi semakin menantang pada dimensi tinggi, terutama bagi algoritma KMA yang sebelumnya menunjukkan performa sangat baik pada dimensi rendah dan menengah. Namun, KMA mengalami degradasi signifikan pada dimensi 200 dan 500, menandakan penurunan stabilitas dan kemampuan adaptasi. Sebaliknya, EHO menunjukkan performa yang lebih stabil dan konsisten dari dimensi rendah hingga tinggi, dengan peningkatan nilai solusi yang jauh lebih moderat. Hal ini menunjukkan bahwa pada fungsi Levy, EHO lebih unggul dalam hal kestabilan serta adaptabilitas terhadap peningkatan dimensi, sementara KMA meskipun lebih baik di dimensi rendah, mulai kehilangan konsistensinya saat dimensi meningkat drastis.

\begin{figure}[H]
\centering
\begin{tikzpicture}
  \begin{axis}[
    title={Hasil \textit{benchmarking} fungsi Rastrigin},
    legend entries={{KMA Best}, {KMA Worst}, {KMA Avg}, {EHO Best}, {EHO Worst}, {EHO Avg}},
    legend pos = outer north east,
    xlabel = Dimensi,
    ylabel = Nilai solusi,
    symbolic x coords={1,2,5,10,20,25,50,100,200,500},
    xtick=data,
    x tick label style={rotate=45, anchor=east},
    enlargelimits=0.05
    ]
    
    % KMA Best
    \addplot[blue,mark=diamond*] table[
      col sep=tab,
      skip first n=1,
      x index=1,
      y index=2,
    ] {data/kma_rastrigin.tsv};
    
    % KMA Worst
    \addplot[green,mark=diamond*] table[
      col sep=tab,
      skip first n=1,
      x index=1,
      y index=3,
    ] {data/kma_rastrigin.tsv};
    
    % KMA Avg
    \addplot[orange,mark=diamond*] table[
      col sep=tab,
      skip first n=1,
      x index=1,
      y index=4,
    ] {data/kma_rastrigin.tsv};
    
    % EHO Best
    \addplot[red,mark=square*] table[
      col sep=tab,
      skip first n=1,
      x index=1,
      y index=2,
    ] {data/eho_rastrigin.tsv};
    
    % EHO Worst
    \addplot[magenta,mark=square*] table[
      col sep=tab,
      skip first n=1,
      x index=1,
      y index=3,
    ] {data/eho_rastrigin.tsv};
    
    % EHO Avg
    \addplot[cyan,mark=square*] table[
      col sep=tab,
      skip first n=1,
      x index=1,
      y index=4,
    ] {data/eho_rastrigin.tsv};
    
  \end{axis}
\end{tikzpicture}
\caption{Grafik hasil \textit{benchmarking} fungsi Rastrigin}
\label{fig:graph_rastrigin}
\end{figure}

Fungsi Rastrigin terbukti sangat multimodal dan menantang untuk stabilitas algoritma, khususnya pada dimensi menengah hingga tinggi. KMA menunjukkan performa yang sangat baik pada nilai \textit{Best} dan \textit{Avg}, namun justru sangat tidak stabil pada skenario \textit{Worst}, terutama pada dimensi 50. Di sisi lain, EHO menunjukkan kestabilan yang lebih baik, khususnya pada dimensi menengah, meskipun performa terburuknya meningkat pada dimensi ekstrem (500). Secara keseluruhan, EHO lebih stabil, sedangkan KMA lebih akurat tetapi berisiko menghasilkan outlier ekstrem.

\begin{figure}[H]
\centering
\begin{tikzpicture}
  \begin{axis}[
    title={Hasil \textit{benchmarking} fungsi Rosenbrock},
    legend entries={{KMA Best}, {KMA Worst}, {KMA Avg}, {EHO Best}, {EHO Worst}, {EHO Avg}},
    legend pos = outer north east,
    xlabel = Dimensi,
    ylabel = Nilai solusi,
    symbolic x coords={1,2,5,10,20,25,50,100,200,500},
    xtick=data,
    x tick label style={rotate=45, anchor=east},
    enlargelimits=0.05
    ]
    
    % KMA Best
    \addplot[blue,mark=diamond*] table[
      col sep=tab,
      skip first n=1,
      x index=1,
      y index=2,
    ] {data/kma_rosenbrock.tsv};
    
    % KMA Worst
    \addplot[green,mark=diamond*] table[
      col sep=tab,
      skip first n=1,
      x index=1,
      y index=3,
    ] {data/kma_rosenbrock.tsv};
    
    % KMA Avg
    \addplot[orange,mark=diamond*] table[
      col sep=tab,
      skip first n=1,
      x index=1,
      y index=4,
    ] {data/kma_rosenbrock.tsv};
    
    % EHO Best
    \addplot[red,mark=square*] table[
      col sep=tab,
      skip first n=1,
      x index=1,
      y index=2,
    ] {data/eho_rosenbrock.tsv};
    
    % EHO Worst
    \addplot[magenta,mark=square*] table[
      col sep=tab,
      skip first n=1,
      x index=1,
      y index=3,
    ] {data/eho_rosenbrock.tsv};
    
    % EHO Avg
    \addplot[cyan,mark=square*] table[
      col sep=tab,
      skip first n=1,
      x index=1,
      y index=4,
    ] {data/eho_rosenbrock.tsv};
    
  \end{axis}
\end{tikzpicture}
\caption{Grafik hasil \textit{benchmarking} fungsi Rosenbrock}
\label{fig:graph_rosenbrock}
\end{figure}

Fungsi Rosenbrock menjadi semakin sulit dioptimasi seiring bertambahnya dimensi. KMA memiliki performa terbaik pada \textit{Best}, tetapi sangat tidak stabil pada \textit{Worst}, terutama pada dimensi tinggi. EHO lebih stabil secara keseluruhan, walaupun tidak selalu memberikan nilai \textit{Best} paling rendah. Dalam konteks fungsi Rosenbrock, konsistensi EHO membuatnya lebih andal pada skenario berdimensi besar, sementara KMA cenderung tidak \textit{robust} terhadap peningkatan kompleksitas.

\begin{figure}[H]
\centering
\begin{tikzpicture}
  \begin{axis}[
    title={Hasil \textit{benchmarking} fungsi Schwefel 1.2 problem},
    legend entries={{KMA Best}, {KMA Worst}, {KMA Avg}, {EHO Best}, {EHO Worst}, {EHO Avg}},
    legend pos = outer north east,
    xlabel = Dimensi,
    ylabel = Nilai solusi,
    symbolic x coords={1,2,5,10,20,25,50,100,200,500},
    xtick=data,
    x tick label style={rotate=45, anchor=east},
    enlargelimits=0.05
    ]
    
    % KMA Best
    \addplot[blue,mark=diamond*] table[
      col sep=tab,
      skip first n=1,
      x index=1,
      y index=2,
    ] {data/kma_schwefel_1_2_problem.tsv};
    
    % KMA Worst
    \addplot[green,mark=diamond*] table[
      col sep=tab,
      skip first n=1,
      x index=1,
      y index=3,
    ] {data/kma_schwefel_1_2_problem.tsv};
    
    % KMA Avg
    \addplot[orange,mark=diamond*] table[
      col sep=tab,
      skip first n=1,
      x index=1,
      y index=4,
    ] {data/kma_schwefel_1_2_problem.tsv};
    
    % EHO Best
    \addplot[red,mark=square*] table[
      col sep=tab,
      skip first n=1,
      x index=1,
      y index=2,
    ] {data/eho_schwefel_1_2_problem.tsv};
    
    % EHO Worst
    \addplot[magenta,mark=square*] table[
      col sep=tab,
      skip first n=1,
      x index=1,
      y index=3,
    ] {data/eho_schwefel_1_2_problem.tsv};
    
    % EHO Avg
    \addplot[cyan,mark=square*] table[
      col sep=tab,
      skip first n=1,
      x index=1,
      y index=4,
    ] {data/eho_schwefel_1_2_problem.tsv};
    
  \end{axis}
\end{tikzpicture}
\caption{Grafik hasil \textit{benchmarking} fungsi Schwefel 1.2 problem}
\label{fig:graph_schwefel_1_2_problem}
\end{figure}

Fungsi Schwefel 1.2 problem menunjukkan karakteristik sensitif terhadap peningkatan dimensi, terutama pada metrik \textit{Worst} dan \textit{Avg}. KMA bekerja sangat baik pada dimensi kecil-menengah, tetapi tidak stabil pada dimensi tinggi, khususnya pada kondisi \textit{Worst}. EHO lebih stabil dalam \textit{Best}, tetapi pada dimensi tinggi \textit{Avg} dan \textit{Worst}-nya tetap mengalami degradasi. Pada dimensi 500, EHO lebih unggul dibanding KMA secara keseluruhan, terutama karena \textit{Worst}-nya lebih rendah dan \textit{Best} tetap stabil.

\begin{figure}[H]
\centering
\begin{tikzpicture}
  \begin{axis}[
    title={Hasil \textit{benchmarking} fungsi Schwefel 2.13 problem},
    legend entries={{KMA Best}, {KMA Worst}, {KMA Avg}, {EHO Best}, {EHO Worst}, {EHO Avg}},
    legend pos = outer north east,
    xlabel = Dimensi,
    ylabel = Nilai solusi,
    symbolic x coords={1,2,5,10,20,25,50,100,200,500},
    xtick=data,
    x tick label style={rotate=45, anchor=east},
    enlargelimits=0.05
    ]
    
    % KMA Best
    \addplot[blue,mark=diamond*] table[
      col sep=tab,
      skip first n=1,
      x index=1,
      y index=2,
    ] {data/kma_schwefel_2_13_problem.tsv};
    
    % KMA Worst
    \addplot[green,mark=diamond*] table[
      col sep=tab,
      skip first n=1,
      x index=1,
      y index=3,
    ] {data/kma_schwefel_2_13_problem.tsv};
    
    % KMA Avg
    \addplot[orange,mark=diamond*] table[
      col sep=tab,
      skip first n=1,
      x index=1,
      y index=4,
    ] {data/kma_schwefel_2_13_problem.tsv};
    
    % EHO Best
    \addplot[red,mark=square*] table[
      col sep=tab,
      skip first n=1,
      x index=1,
      y index=2,
    ] {data/eho_schwefel_2_13_problem.tsv};
    
    % EHO Worst
    \addplot[magenta,mark=square*] table[
      col sep=tab,
      skip first n=1,
      x index=1,
      y index=3,
    ] {data/eho_schwefel_2_13_problem.tsv};
    
    % EHO Avg
    \addplot[cyan,mark=square*] table[
      col sep=tab,
      skip first n=1,
      x index=1,
      y index=4,
    ] {data/eho_schwefel_2_13_problem.tsv};
    
  \end{axis}
\end{tikzpicture}
\caption{Grafik hasil \textit{benchmarking} fungsi Schwefel 2.13 problem}
\label{fig:graph_schwefel_2_13_problem}
\end{figure}

Fungsi Schwefel 2.13 problem tergolong mudah untuk dioptimasi, bahkan pada dimensi tinggi. Dalam kasus ini, KMA dan EHO menunjukkan performa yang sangat mirip, dengan nilai solusi yang rendah dan tren peningkatan yang seimbang. Baik nilai \textit{Best}, \textit{Avg}, maupun \textit{Worst} dari kedua algoritma tetap dalam kisaran kecil (kurang dari 1.5), menunjukkan tingkat kestabilan dan akurasi yang tinggi. Ini berarti tidak ada algoritma yang secara signifikan lebih unggul, keduanya sama-sama efektif dan efisien dalam menyelesaikan fungsi ini.

\begin{figure}[H]
\centering
\begin{tikzpicture}
  \begin{axis}[
    title={Hasil \textit{benchmarking} fungsi Shubert},
    legend entries={{KMA Best}, {KMA Worst}, {KMA Avg}, {EHO Best}, {EHO Worst}, {EHO Avg}},
    legend pos = outer north east,
    xlabel = Dimensi,
    ylabel = Nilai solusi,
    symbolic x coords={1,2,5,10,20,25,50,100,200,500},
    xtick=data,
    x tick label style={rotate=45, anchor=east},
    enlargelimits=0.05
    ]
    
    % KMA Best
    \addplot[blue,mark=diamond*] table[
      col sep=tab,
      skip first n=1,
      x index=1,
      y index=2,
    ] {data/kma_shubert.tsv};
    
    % KMA Worst
    \addplot[green,mark=diamond*] table[
      col sep=tab,
      skip first n=1,
      x index=1,
      y index=3,
    ] {data/kma_shubert.tsv};
    
    % KMA Avg
    \addplot[orange,mark=diamond*] table[
      col sep=tab,
      skip first n=1,
      x index=1,
      y index=4,
    ] {data/kma_shubert.tsv};
    
    % EHO Best
    \addplot[red,mark=square*] table[
      col sep=tab,
      skip first n=1,
      x index=1,
      y index=2,
    ] {data/eho_shubert.tsv};
    
    % EHO Worst
    \addplot[magenta,mark=square*] table[
      col sep=tab,
      skip first n=1,
      x index=1,
      y index=3,
    ] {data/eho_shubert.tsv};
    
    % EHO Avg
    \addplot[cyan,mark=square*] table[
      col sep=tab,
      skip first n=1,
      x index=1,
      y index=4,
    ] {data/eho_shubert.tsv};
    
  \end{axis}
\end{tikzpicture}
\caption{Grafik hasil \textit{benchmarking} fungsi Shubert}
\label{fig:graph_shubert}
\end{figure}

Kedua algoritma menunjukkan stabilitas dalam menyelesaikan fungsi Shubert pada dimensi rendah hingga sedang. Namun, terjadi error numerik pada EHO Best di dimensi tinggi (500), yang menghasilkan nilai negatif besar akibat \textit{overflow}. Sementara itu, KMA menunjukkan stabilitas yang lebih baik dalam menangani fungsi Shubert berdimensi tinggi. \cref{fig:graph_shubert} menegaskan perlunya penanganan numerik seperti \textit{clipping}, \textit{casting}, atau limitasi domain pada fungsi Shubert untuk dimensi besar agar dapat menghindari \textit{overflow} saat \textit{benchmarking}.

\begin{figure}[H]
\centering
\begin{tikzpicture}
  \begin{axis}[
    title={Hasil \textit{benchmarking} fungsi Sphere},
    legend entries={{KMA Best}, {KMA Worst}, {KMA Avg}, {EHO Best}, {EHO Worst}, {EHO Avg}},
    legend pos = outer north east,
    xlabel = Dimensi,
    ylabel = Nilai solusi,
    symbolic x coords={1,2,5,10,20,25,50,100,200,500},
    xtick=data,
    x tick label style={rotate=45, anchor=east},
    enlargelimits=0.05
    ]
    
    % KMA Best
    \addplot[blue,mark=diamond*] table[
      col sep=tab,
      skip first n=1,
      x index=1,
      y index=2,
    ] {data/kma_sphere.tsv};
    
    % KMA Worst
    \addplot[green,mark=diamond*] table[
      col sep=tab,
      skip first n=1,
      x index=1,
      y index=3,
    ] {data/kma_sphere.tsv};
    
    % KMA Avg
    \addplot[orange,mark=diamond*] table[
      col sep=tab,
      skip first n=1,
      x index=1,
      y index=4,
    ] {data/kma_sphere.tsv};
    
    % EHO Best
    \addplot[red,mark=square*] table[
      col sep=tab,
      skip first n=1,
      x index=1,
      y index=2,
    ] {data/eho_sphere.tsv};
    
    % EHO Worst
    \addplot[magenta,mark=square*] table[
      col sep=tab,
      skip first n=1,
      x index=1,
      y index=3,
    ] {data/eho_sphere.tsv};
    
    % EHO Avg
    \addplot[cyan,mark=square*] table[
      col sep=tab,
      skip first n=1,
      x index=1,
      y index=4,
    ] {data/eho_sphere.tsv};
    
  \end{axis}
\end{tikzpicture}
\caption{Grafik hasil \textit{benchmarking} fungsi Sphere}
\label{fig:graph_sphere}
\end{figure}

Fungsi Sphere merupakan fungsi konveks sederhana dan mudah dioptimasi, di mana kedua algoritma (KMA dan EHO) bekerja sangat baik dalam menemukan solusi optimal terutama pada \textit{Best} dan \textit{Avg} di dimensi rendah, dengan KMA lebih stabil pada semua dimensi terutama dalam mengontrol kasus \textit{Worst}, sementara EHO menunjukkan penurunan performa pada dimensi tinggi dalam kasus \textit{Worst} namun tetap kompetitif pada \textit{Best} dan \textit{Avg}.

\begin{figure}[H]
\centering
\begin{tikzpicture}
  \begin{axis}[
    title={Hasil \textit{benchmarking} fungsi Step function},
    legend entries={{KMA Best}, {KMA Worst}, {KMA Avg}, {EHO Best}, {EHO Worst}, {EHO Avg}},
    legend pos = outer north east,
    xlabel = Dimensi,
    ylabel = Nilai solusi,
    symbolic x coords={1,2,5,10,20,25,50,100,200,500},
    xtick=data,
    x tick label style={rotate=45, anchor=east},
    enlargelimits=0.05
    ]
    
    % KMA Best
    \addplot[blue,mark=diamond*] table[
      col sep=tab,
      skip first n=1,
      x index=1,
      y index=2,
    ] {data/kma_step_function.tsv};
    
    % KMA Worst
    \addplot[green,mark=diamond*] table[
      col sep=tab,
      skip first n=1,
      x index=1,
      y index=3,
    ] {data/kma_step_function.tsv};
    
    % KMA Avg
    \addplot[orange,mark=diamond*] table[
      col sep=tab,
      skip first n=1,
      x index=1,
      y index=4,
    ] {data/kma_step_function.tsv};
    
    % EHO Best
    \addplot[red,mark=square*] table[
      col sep=tab,
      skip first n=1,
      x index=1,
      y index=2,
    ] {data/eho_step_function.tsv};
    
    % EHO Worst
    \addplot[magenta,mark=square*] table[
      col sep=tab,
      skip first n=1,
      x index=1,
      y index=3,
    ] {data/eho_step_function.tsv};
    
    % EHO Avg
    \addplot[cyan,mark=square*] table[
      col sep=tab,
      skip first n=1,
      x index=1,
      y index=4,
    ] {data/eho_step_function.tsv};
    
  \end{axis}
\end{tikzpicture}
\caption{Grafik hasil \textit{benchmarking} fungsi Step function}
\label{fig:graph_step_function}
\end{figure}

Fungsi Step sangat mudah dioptimasi oleh kedua algoritma (KMA dan EHO), bahkan pada dimensi sangat tinggi (500), tanpa menimbulkan tantangan berarti terhadap proses pencarian solusi baik dari segi akurasi maupun kestabilan, dan tidak ditemukan perbedaan performa antara KMA dan EHO karena keduanya mampu mencapai hasil optimal pada semua kondisi.

\begin{figure}[H]
\centering
\begin{tikzpicture}
  \begin{axis}[
    title={Hasil \textit{benchmarking} fungsi Vincent},
    legend entries={{KMA Best}, {KMA Worst}, {KMA Avg}, {EHO Best}, {EHO Worst}, {EHO Avg}},
    legend pos = outer north east,
    xlabel = Dimensi,
    ylabel = Nilai solusi,
    symbolic x coords={1,2,5,10,20,25,50,100,200,500},
    xtick=data,
    x tick label style={rotate=45, anchor=east},
    enlargelimits=0.05
    ]
    
    % KMA Best
    \addplot[blue,mark=diamond*] table[
      col sep=tab,
      skip first n=1,
      x index=1,
      y index=2,
    ] {data/kma_vincent.tsv};
    
    % KMA Worst
    \addplot[green,mark=diamond*] table[
      col sep=tab,
      skip first n=1,
      x index=1,
      y index=3,
    ] {data/kma_vincent.tsv};
    
    % KMA Avg
    \addplot[orange,mark=diamond*] table[
      col sep=tab,
      skip first n=1,
      x index=1,
      y index=4,
    ] {data/kma_vincent.tsv};
    
    % EHO Best
    \addplot[red,mark=square*] table[
      col sep=tab,
      skip first n=1,
      x index=1,
      y index=2,
    ] {data/eho_vincent.tsv};
    
    % EHO Worst
    \addplot[magenta,mark=square*] table[
      col sep=tab,
      skip first n=1,
      x index=1,
      y index=3,
    ] {data/eho_vincent.tsv};
    
    % EHO Avg
    \addplot[cyan,mark=square*] table[
      col sep=tab,
      skip first n=1,
      x index=1,
      y index=4,
    ] {data/eho_vincent.tsv};
    
  \end{axis}
\end{tikzpicture}
\caption{Grafik hasil \textit{benchmarking} fungsi Vincent}
\label{fig:graph_vincent}
\end{figure}

Fungsi Vincent mampu menguji kestabilan algoritma terhadap dimensi dan \textit{noise} dalam hasil pencarian, di mana KMA menunjukkan performa yang lebih stabil dan konsisten dibandingkan EHO pada seluruh dimensi, sementara EHO memiliki kemampuan eksplorasi yang kuat dengan \textit{Best} mendekati -1 namun kurang konsisten pada nilai \textit{Avg} dan \textit{Worst}, sehingga secara keseluruhan KMA lebih unggul terutama dalam kestabilan dan ketahanan terhadap peningkatan dimensi.

\begin{figure}[H]
\centering
\begin{tikzpicture}
  \begin{axis}[
    title={Hasil \textit{benchmarking} fungsi Weierstrass},
    legend entries={{KMA Best}, {KMA Worst}, {KMA Avg}, {EHO Best}, {EHO Worst}, {EHO Avg}},
    legend pos = outer north east,
    xlabel = Dimensi,
    ylabel = Nilai solusi,
    symbolic x coords={1,2,5,10,20,25,50,100,200,500},
    xtick=data,
    x tick label style={rotate=45, anchor=east},
    enlargelimits=0.05
    ]
    
    % KMA Best
    \addplot[blue,mark=diamond*] table[
      col sep=tab,
      skip first n=1,
      x index=1,
      y index=2,
    ] {data/kma_weierstrass.tsv};
    
    % KMA Worst
    \addplot[green,mark=diamond*] table[
      col sep=tab,
      skip first n=1,
      x index=1,
      y index=3,
    ] {data/kma_weierstrass.tsv};
    
    % KMA Avg
    \addplot[orange,mark=diamond*] table[
      col sep=tab,
      skip first n=1,
      x index=1,
      y index=4,
    ] {data/kma_weierstrass.tsv};
    
    % EHO Best
    \addplot[red,mark=square*] table[
      col sep=tab,
      skip first n=1,
      x index=1,
      y index=2,
    ] {data/eho_weierstrass.tsv};
    
    % EHO Worst
    \addplot[magenta,mark=square*] table[
      col sep=tab,
      skip first n=1,
      x index=1,
      y index=3,
    ] {data/eho_weierstrass.tsv};
    
    % EHO Avg
    \addplot[cyan,mark=square*] table[
      col sep=tab,
      skip first n=1,
      x index=1,
      y index=4,
    ] {data/eho_weierstrass.tsv};
    
  \end{axis}
\end{tikzpicture}
\caption{Grafik hasil \textit{benchmarking} fungsi Weierstrass}
\label{fig:graph_weierstrass}
\end{figure}

Fungsi Weierstrass menunjukkan peningkatan kesulitan optimasi yang signifikan seiring pertambahan dimensi, terutama bagi algoritma EHO. KMA terbukti lebih stabil dengan nilai \textit{Avg} yang tetap rendah, meskipun terdapat variasi signifikan pada kondisi \textit{Worst}. Di sisi lain, EHO menunjukkan sensitivitas tinggi terhadap kenaikan dimensi, di mana nilai \textit{Avg} dan \textit{Worst} meningkat tajam mengindikasikan potensi masalah dalam eksploitasi atau proses konvergensi.

\begin{figure}[H]
\centering
\begin{tikzpicture}
  \begin{axis}[
    title={Hasil \textit{benchmarking} fungsi Zakharov},
    legend entries={{KMA Best}, {KMA Worst}, {KMA Avg}, {EHO Best}, {EHO Worst}, {EHO Avg}},
    legend pos = outer north east,
    xlabel = Dimensi,
    ylabel = Nilai solusi,
    symbolic x coords={1,2,5,10,20,25,50,100,200,500},
    xtick=data,
    x tick label style={rotate=45, anchor=east},
    enlargelimits=0.05
    ]
    
    % KMA Best
    \addplot[blue,mark=diamond*] table[
      col sep=tab,
      skip first n=1,
      x index=1,
      y index=2,
    ] {data/kma_zakharov.tsv};
    
    % KMA Worst
    \addplot[green,mark=diamond*] table[
      col sep=tab,
      skip first n=1,
      x index=1,
      y index=3,
    ] {data/kma_zakharov.tsv};
    
    % KMA Avg
    \addplot[orange,mark=diamond*] table[
      col sep=tab,
      skip first n=1,
      x index=1,
      y index=4,
    ] {data/kma_zakharov.tsv};
    
    % EHO Best
    \addplot[red,mark=square*] table[
      col sep=tab,
      skip first n=1,
      x index=1,
      y index=2,
    ] {data/eho_zakharov.tsv};
    
    % EHO Worst
    \addplot[magenta,mark=square*] table[
      col sep=tab,
      skip first n=1,
      x index=1,
      y index=3,
    ] {data/eho_zakharov.tsv};
    
    % EHO Avg
    \addplot[cyan,mark=square*] table[
      col sep=tab,
      skip first n=1,
      x index=1,
      y index=4,
    ] {data/eho_zakharov.tsv};
    
  \end{axis}
\end{tikzpicture}
\caption{Grafik hasil \textit{benchmarking} fungsi Zakharov}
\label{fig:graph_zakharov}
\end{figure}

Pada fungsi Zakharov, kedua algoritma menunjukkan kestabilan hingga dimensi 200, dengan peningkatan kecil pada nilai \textit{best} EHO dan nilai \textit{avg} KMA. Namun, KMA mengalami degradasi performa yang signifikan pada kondisi terburuk saat dimensi mencapai 500, ditandai dengan lonjakan drastis pada nilai \textit{worst}. Meskipun demikian, kedua algoritma masih mampu mempertahankan nilai \textit{best} yang rendah, menunjukkan kemampuan keduanya dalam menemukan solusi optimal. Secara keseluruhan, EHO lebih konsisten dan stabil pada fungsi Zakharov, terutama pada dimensi tinggi, sementara KMA tetap cocok digunakan pada dimensi rendah hingga sedang.

\begin{table}[h!]
    \centering
    \caption{rangkuman hasil \textit{benchmarking}}
    \begin{tabular}{|l|l|l|l|l|}
    \hline
        & best & worst & avg & std \\ \hline
        EHO & 42 & 149 & 108 & 144 \\ \hline
        KMA & 163 & 60 & 100 & 63 \\ \hline
        seri & 14 & 11 & 11 & 11 \\ \hline
        invalid & 1 & 0 & 1 & 2 \\ \hline
        total & 220 & 220 & 220 & 220 \\ \hline
    \end{tabular}
    \label{summary-result}
\end{table}

\section{Pengujian}
\subsection{\textit{Unit test}}
Tahap awal pengujian dilakukan melalui implementasi \textit{unit test} untuk memverifikasi kinerja fungsi \textit{benchmark}. Pengujian ini mengevaluasi kemampuan fungsi menghasilkan \textit{output} sesuai ekspektasi dengan berbagai parameter, termasuk variasi dimensi untuk menguji skalabilitas, beragam nilai input untuk konsistensi kalkulasi, nilai rotasi dan \textit{shift} untuk mengukur invariansi transformasi, serta nilai bias yang bervariasi. Unit test ini fundamental untuk memastikan keandalan dan akurasi fungsi \textit{benchmark} sebelum digunakan dalam optimasi kompleks, memungkinkan identifikasi dini kesalahan perhitungan atau inkonsistensi perilaku. Dengan demikian, integritas fungsi \textit{benchmark} terjamin sebagai fondasi yang kokoh untuk analisis performa algoritma optimasi.

\textit{Method setup} berfungsi untuk menginisialisasi properti yang akan digunakan pada \textit{test case}. Properti yang diinisialisasi yaitu seperti dimensi untuk fungsi CEC dan COCO, rotasi, \textit{shift}, \textit{bias} untuk fungsi CEC seperti pada \cref{lst:setup_test_case_cec}, dan \textit{seed} untuk fungsi COCO pada \cref{lst:setup_test_case_coco}
\begin{lstlisting}[language=Python, caption=\textit{setup unit test} fungsi CEC, label=lst:setup_test_case_cec]
    def setUp(self):
        # Example rotation matrix and shift vector for testing
        # Identity matrix (no rotation)
        self.rotation_identity = np.eye(2)
        self.rotation_non_identity = np.array([[0, -1], [1, 0]])
        self.shift = np.array([1, 1])
        self.no_shift = np.zeros(2)
        self.f_bias = 1.0
        self.dimension = 2
\end{lstlisting}
\begin{lstlisting}[language=Python, caption=\textit{setup unit test} fungsi COCO, label=lst:setup_test_case_coco]
    def setUp(self):
        np.random.seed(42)
\end{lstlisting}
\textit{Test case} yang dirancang mencakup berbagai skenario untuk menguji fungsi benchmark dalam kondisi yang beragam. Setiap \textit{test case} mengevaluasi aspek spesifik dari fungsi untuk memastikan perilaku yang konsisten dan akurat.

Pengujian tanpa transformasi ditunjukkan pada \textit{test case} pertama (\cref{lst:no_shift_no_rotation}) yang menggunakan matriks rotasi identitas dan tanpa \textit{shift}. Pada \cref{lst:no_shift_no_rotation} menguji fungsi Sphere dengan \textit{input} $[3.0, 2.0]$ dan memverifikasi bahwa hasil evaluasi sesuai dengan perhitungan manual yaitu 13.0. \textit{Test case} ini memastikan fungsi bekerja dengan benar dalam kondisi paling sederhana tanpa transformasi apapun.
\begin{lstlisting}[language=Python, caption=\textit{Test case} tanpa rotasi dan \textit{shift}, label=lst:no_shift_no_rotation]
    def test_evaluate_with_identity_rotation_and_no_shift(self):
        sphere = Sphere(self.dimension, self.rotation_identity, self.no_shift)
        input_vector = np.array([3.0, 2.0])
        result = sphere.evaluate(input_vector)
        expected_result = 13.0
        self.assertAlmostEqual(result, expected_result, places=5)
\end{lstlisting}
Pengujian dengan transformasi lengkap didemonstrasikan pada \textit{test case} kedua (\cref{lst:shift_rotation}) yang menerapkan rotasi non-identitas dan \textit{shift}. Menggunakan \textit{input} yang sama $[3.0, 2.0]$, \textit{test case} ini memverifikasi bahwa transformasi rotasi dan \textit{shift} berfungsi dengan benar dengan hasil yang diharapkan 5.0. Ini menguji kemampuan fungsi dalam menangani transformasi ruang pencarian yang kompleks.
\begin{lstlisting}[language=Python, caption=\textit{Test case} dengan rotasi dan \textit{shift}, label=lst:shift_rotation]
    def test_evaluate_with_non_identity_rotation_and_shift(self):
        sphere = Sphere(self.dimension, self.rotation_non_identity, self.shift)
        input_vector = np.array([3.0, 2.0])
        result = sphere.evaluate(input_vector)
        expected_result = 5.0
        self.assertAlmostEqual(result, expected_result, places=5)
\end{lstlisting}
\textit{Test case} untuk bias mengevaluasi pengaruh parameter $f_{\text{bias}}$ terhadap hasil akhir fungsi. Dengan menggunakan \textit{input} vector nol dan menambahkan nilai bias, \textit{test case} ini memastikan bahwa bias diterapkan dengan benar sebagai konstanta additif pada hasil evaluasi fungsi.
\begin{lstlisting}[language=Python, caption=\textit{Test case} dengan \textit{bias}, label=lst:bias]
    def test_evaluate_with_f_bias(self):
        sphere = Sphere(self.dimension, self.rotation_identity,
                        self.no_shift, self.f_bias)
        input_vector = np.zeros(2)
        result = sphere.evaluate(input_vector)
        expected_result = 0.0 + self.f_bias
        self.assertAlmostEqual(result, expected_result, places=5)
\end{lstlisting}
Pengujian dengan input khusus mencakup dua skenario: input bernilai nol (\cref{lst:nol}) dan input bernilai negatif (\cref{lst:negative}). \textit{Test case} untuk input nol menggunakan fungsi Attractive sector yang lebih kompleks dengan transformasi multi-tahap meliputi matriks $Q$, diagonal matrix ($\Lambda^{\alpha}$), dan matriks $R$, diikuti dengan transformasi $T_{\text{osz}}$ dan penambahan $f_{\text{bias}}$. Test case untuk input negatif menggunakan skenario serupa dengan \textit{input} $[-3, -2]$ untuk memastikan fungsi dapat menangani nilai negatif dengan benar. Kedua \textit{test case} ini menghitung hasil yang diharapkan secara manual dan membandingkannya dengan output fungsi untuk memverifikasi akurasi implementasi.
\begin{lstlisting}[language=Python, caption=\textit{Test case} dengan input bernilai nol, label=lst:nol]
    def test_evaluate_at_zero_vector(self):
        """
        Test the evaluate method at the zero vector.
        """
        dimension = 2
        test_func = Attractive_sector(dimension)
        benchmark = Benchmark(dimension)

        # Zero vector
        input_vector = np.zeros(dimension)
        result = test_func.evaluate(input_vector)

        # Manually compute the expected result
        z = test_func.Q @ test_func.diag_matrix @ test_func.R @ (
            input_vector - test_func.x_opt)
        s = np.where((z * test_func.x_opt), 10 ** 2, 1)

        expected_result = np.sum((s * z) ** 2)
        expected_result = benchmark.T_osz(
            expected_result ** 0.9) + test_func.f_opt

        self.assertAlmostEqual(result, expected_result, places=6)
\end{lstlisting}
\begin{lstlisting}[language=Python, caption=\textit{Test case} dengan input bernilai negatif, label=lst:negative]
    def test_evaluate_with_negative_values(self):
        dimension = 2
        test_func = Attractive_sector(dimension)
        benchmark = Benchmark(dimension)

        # Input vector with negative values
        input_vector = np.array([-3, -2])
        result = test_func.evaluate(input_vector)

        # Manually compute the expected result
        z = test_func.Q @ test_func.diag_matrix @ test_func.R @ (
            input_vector - test_func.x_opt)
        s = np.where((z * test_func.x_opt), 10 ** 2, 1)

        expected_result = np.sum((s * z) ** 2)
        expected_result = benchmark.T_osz(
            expected_result ** 0.9) + test_func.f_opt

        self.assertAlmostEqual(result, expected_result, places=6)
\end{lstlisting}

\subsection{\textit{Compability test}}
Setelah seluruh fungsi diuji secara individual melalui \textit{unit test}, tahap berikutnya adalah pengujian kompatibilitas \textit{library} CECO terhadap berbagai versi Python dan NumPy. Tujuannya adalah memastikan bahwa \textit{library} dapat berjalan stabil dan konsisten di berbagai konfigurasi sistem yang umum digunakan oleh pengguna.

Untuk mendukung proses ini, digunakan Docker sebagai \textit{platform} utama pengujian. Docker memungkinkan pembuatan lingkungan terisolasi (\textit{container}) yang dapat dikonfigurasi dengan versi Python dan NumPy tertentu tanpa mengganggu sistem utama. Pendekatan ini tidak hanya mengurangi risiko konflik dependensi, tetapi juga mempercepat proses pengujian karena versi lingkungan dapat diatur secara fleksibel dan otomatis.

Pada \cref{lst:dockerfile} baris ke 1-2 menentukan \textit{base image} yang digunakan. Dengan menggunakan argumen BASE\_IMAGE, pengujian bisa dilakukan pada berbagai versi python, seperti python:3.6, python:3.8, python:3.10, dan versi lainnya. Pada \cref{lst:dockerfile} baris ke 4 merupakan perintah yang memastikan bahwa \textit{tools} penting untuk kompilasi dan instalasi tersedia di dalam kontainer. Pada \cref{lst:dockerfile} baris ke 6-8 akan menyalin seluruh isi direktori proyek ke dalam direktori kerja kontainer, yakni /app. Pada \cref{lst:dockerfile} baris ke 10-11 akan menggunakan argumen NUMPY\_VERSION untuk menginstal versi numpy tertentu, misalnya numpy==1.19.0 atau numpy==2.2, sesuai dengan konfigurasi pengujian. Pada \cref{lst:dockerfile} baris ke 13 akan menginstal \textit{library} CECO dalam mode \textit{editable}, sehingga perubahan lokal pada kode dapat langsung diuji. Pada \cref{lst:dockerfile} baris ke 15 akan mengeksekusi skrip pengujian \textit{unit test} setelah \textit{environment} siap.
\lstinputlisting[caption=kode docker file, label=lst:dockerfile]{kode/dockerfile}

Untuk mem\textit{build} dan menjalankan dockerfile tersebut digunakan perintah pada \textit{terminal} dengan \cref{lst:docker_build}. Pada \cref{lst:docker_build} baris pertama membuat \textit{image} baru bernama ceco-test:py39-np22 dengan python 3.9 dan numpy versi 2.2. Pada \cref{lst:docker_build} baris kedua menjalankan container berdasarkan \textit{image} yang telah dibuat, secara otomatis menjalankan pengujian pada lingkungan yang telah dikonfigurasi.

\begin{lstlisting}[caption=Perintah untuk build dockerfile dan menjalankan container hasil build, label=lst:docker_build]
docker build --build-arg BASE_IMAGE=python:3.9 --build-arg NUMPY_VERSION="numpy==2.2" -t ceco-test:py39-np22 .
docker run ceco-test:py39-np22
\end{lstlisting}


Dengan menjalankan variasi \textit{build} semacam ini untuk berbagai kombinasi versi python dan numPy, pengembang dapat dengan cepat mengidentifikasi versi minimum yang didukung, serta memastikan tidak terjadi regresi pada versi-versi terbaru. Berdasarkan hasil pengujian ini, diketahui bahwa \textit{library} CECO kompatibel mulai dari Python 3.6 dan NumPy 1.19.0, informasi yang kemudian disertakan dalam dokumentasi resmi sebagai acuan pengguna.